\documentclass[]{article}


% Title Page
\title{Defect Tracking Policy and Workflow}
\author{Michael Ghaben}
\usepackage{listings}
\usepackage{hyperref}

\begin{document}
\maketitle

This section introduces the software tracking, versioning, and tools used by the Low-Comotivation group. This section is broken down into two subsections: 
\begin{enumerate}
	\item A numbering and versioning policy for the software released
	\item A software defect tracking  \& resolution policy 
	\item Development Tools
\end{enumerate}

\subsection{Numbering and Versioning}
Unless otherwise stated, the numbering and versioning of the software utilized throughout this project will be completed by be numbered according to the GitHub commit that is being referenced, until the prototype demonstration. This will be versioned 1.0.

Additionally, the numbering of issues will be chronological and generated by GitHub. Issues will be referred to by the GitHub generated number. This includes bugs and missing features.
\subsection{Defect Tracking and Resolution}
When a defect is observed in the code, a GitHub issue will be raised with the 'bug' tag. After this recording, the individual who observes the bug will mark the bug with a comment of the following form:

\begin{lstlisting}
	\bug {<description of bug>}
	<code block>
\end{lstlisting}

which will automatically document the bug in Doxygen \footnote{We are aware Doxygen is licensed under the LGPL software. This does not include software which utilizes Doxygen to produce documentation} After raising the GitHub issue, a team member will then be assigned to the issue. After the issues successful resolution, the GitHub issue will be closed with a reference comment to the commit which resolves the issue.

\subsection{Development Tools}
During the course of the execution of this project, we will utilize:
\begin{itemize}
	\item GitHub: GitHub will be utilized for version control as well as source control 
	\item Slack: Slack will be utilized for informal discussion among the group members to discuss bugs, planning, and presentation coordination. Web hook integration for GitHub integration and issue tracking will also be utilized.
	\item Travis-CI: Travis-CI will be utilized in conjunction with the Maven build system to provide software testing
	\item Maven: Maven is a cross-platform Java build system that will be used for automated testing and automated building
\end{itemize}
\end{document}          
