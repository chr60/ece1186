\documentclass[]{article}
\usepackage{hyperref}
\usepackage{multirow}
\usepackage{float}
% Title Page
\title{Low-motovation: Group Test Plan}
\author{Michael Ghaben}
\date{}

\begin{document}
\maketitle
\tableofcontents
\titlepage
\section{Introduction}
\subsection{Document Identifier}
This document is the Low-Comotovation Group Test Plan for the design review phase of the Spring 2017 software engineering project.. In this document we detail the testing specifications, requirements, and procedures for the implementation and evaluation of testing procedures.

\subsection{Scope}
In this document a number of assumptions regarding the projects scope and lifecycle are made. Specifically:

\begin{enumerate}
	\item Due to the short development cycle associated with this project, some non-negligble defects will likely persist
	\item Under the iterative development methodology this group is following, the development of additional tests may be completed to address defects is expected
\end{enumerate}
To better address these constraints, we will utilize a continuous integration methodology around iterative development and testing.So that we may ensure rapid development with minimal time overhead, we shall utilize continuous integration tools and methodology. Testing will be broadly divided into two categories: subsystem, and integration. Subsystem testing will be primarily focused on testing an individual system to ensure minimal functionality of a given subsystem and be done primarily independently by each group member. Integration testing will be designated as tests which require two or more modules functionality to satisfy requirements.

Specifically, it is anticipated that software defects will be found and require that testing which was not anticipated. As a result, we shall primarily focus on defining general tsts to be implemented and leave the exact testing requirements as implementation is completed. Additionally, will primarily focus on subsytems to be delivered to the end-user, the train company.

Examples of these subsystems would be the CTC module or the train controller, which are expected to be integrated into the final deliverable. Subsystems which are not examples of the final deliverables are the Train Model and Track Model subsystems, as they will ultimately be removed and replaced with the physical subsystems.

\subsection{System Overview and Key Features}
The purpose of the system under development is a to provide a train system for the Pittsburgh North Shore Rail system.

This system broadly consists of 6 subsystems:
\begin{enumerate}
	\item  Track Model - a physical model of a track to be used for testing
	\item Train Model - a physical model of a train to be used for testing
	\item CTC System - A CTC system to be implemented on the final train system
	\item MBO - ???
	\item Wayside System - A wayside for coordinating with all models
\end{enumerate}
For a further discussion of the system, we refer the reader to the project reuqirements and discussion board.

\section{Test Overview}
The test organization is broadly divided into two sections: subsystem and integration testing. Subsystem tests are regarded as tests associated with only a single system at a time. Integration tests broadly refer to the tests of the integration of more than a single subsystem. In this view, testing is accomplished by each subsystem independently at the discresion of the individual developing the subsystem. Then, as members of the team develop their subsystem, each member shall attempt to integrate and develop tests for their integration. Due to the nature of continuous integration, tests are expected to be developed in parallel to the development of the program modules. 

\subsection{Master Test Schedule}
The test schedul will be implemented as follows::

\subsection{Integrity Level Schema}
For this project, we utilize three integrity levels. The integrity levels for this project are as follows:
\begin{enumerate}
	\item The lowest severity level. This is reserved for tasks which will ultimately not be passed onto the finished product and pose no threat to catastrophic failure. 
	\item This severity level is reserved for failures which may lead to errors in subsystem integration or incorrect information delivered to critical subsystems
	\item The most severe integrity level. Is a vital system or otherwise threatens life or limb in an imlemented subsystem
\end{enumerate}
\subsection{Responsibilities}
Michael Ghaben will be responsible for the integrity of the automated build system as well as the integrity of the master branch.

\subsection{Tools, Techniques, and Metrics}
To implement the test environment and better facilitate a continuous integration test environment, we utilize Travis-CI\footnote{Travis-CI.org} with GitHub\footnote{github.com} integration with Slack\footnote{slack.com} and Jupiter JUnit \footnote{junit.org} integration. By utllizing these tools, we allow automated tests to be run remotely using the Maven\footnote{maven.apache.org}. This allows for rapid feedback into the developmental process, allowing for better integration and testing of the team. The usage of these tools creates feedback loop between implementation, testing, and integration, leading to significant productivity gains. In each case ,testing will be carried out remotely by the build system.

The system shall be quantitatively evaluated on percentage of percentage of test passing. Each subsystem shall be responsible for the determination of importance of testing individual components of their subsystem, with the exception of vital components.

\section{Details}
\subsection{Process}
A test shall be detailed in the following manner:
\begin{table}[H]
	\centering
	\caption{Test Plan}
	\label{my-label}
	\begin{tabular}{|l|l|}
		\hline
		Task & \multicolumn{1}{c|}{Test Design} \\ \hline
		Integrity Level & What is the integrity level? \\hline
		Methods & How? \\ \hline
		Inputs &  What Inputs? \\ \hline
		Outputs &  What is a successful output?\\ \hline
		Expected Completion & When will it be done?\\ \hline
		Risks and Assumptions & What are you assuming?\\ \hline
		Responsibility & Who are you?\\ \hline
	\end{tabular}
\end{table}
In this, we expect to utilize both integration as well as unit tests. After each member pushes to GitHub on his or her respective branch, Travis-CI will provide regression testing on all unit and integration tests that have been implemented. To merge into master, all tests must be passing. 

\subsection{Test Documentation Requirements}
Each test shall be documented using the above table as well as any auxillary information by whomever holds testing responsibilities. Each module shall have a specified test plan for each module covering their individual component for testing. Each subsystem test plan shall detail unit tests for his or her own module. Additionally, integration testing will be accomplished in a similar fashon completed by the group.

Furthermore, the unit and integration testing procedure will be supplemented by functional testing. Each group member shall conduct user testing of each other module. During this time, the testing member will attempt to cause defective behavior at any level. These defects will be tracked via GitHub issues. This testing will occur weekly.

\subsection{Test Administration Requirements}
For a unit or integration test to be considered complete, it must successfully build on the build server utilizing the Maven build system. This ensures consistent repeatable builds to attempt to ensure the clients functional requirements will be met.

Additionally, for functional testing it is expected that each group member submits either a bug report via GitHub issues. Should a group member fail to find any defects and register them, he or she must challenge Professor Profeta to find a defect at the next class meeting. If the professor discovers a defect, the group member(s) who failed to find defects owe the other group members pizza. It is hoped that this procedure will lead to people finding more defects.
\subsection{Test Reporting Requirements}
Each written unit and integrationtest report will be provided by the Maven automated build system. 

To document user testing (e.g. by working with the user interface and attempting to find defects in that manner), a developmer may supplement this test report with the following syntax to automatically document the bug utilizng the following syntax:\newline \newline
/** \newline
* @bug $<Descriptive$ $ message>$ \newline
*/\newline
\newline
This will lead to documentation of the defect in the autmatically generated Doxygen documentation. Note that this should be used for defects which  are not necessarily covered under tests at a given time. By utilizing this process, an iterative cycle of development these defects may be tracked and inclusion in later tests to ensure proper functionality.

\section{Track Module Test Plan}
Author: Michael Ghaben

\subsection{Unit Tests}
\begin{table}[H]
	\centering
	\caption{CSV Reading Test Plan}
	\begin{tabular}{|l|l|}
		\hline
		Task & \multicolumn{1}{c|}{Test Design} \\ \hline
		Integrity Level & 1 \\ \hline
		Methods & Evaluate the readCSV function\\ \hline
		Inputs &  the files redline.csv and greenline.csv \\ \hline
		Outputs &  The track model successfully reading the csv files\\ \hline
		Expected Completion & At initialization of the program\\ \hline
		Risks and Assumptions & Both redline and greenline have been properly input to the files\\ \hline
		Responsibility & Track Model\\ \hline
	\end{tabular}
\end{table}

\subsection{Integration Tests}

\section{Track Controller Test Plan}
\subsection{Unit Tests}

\begin{table}[H]
	\centering
	\caption{Load PLC Program via Browse Button}
	\begin{tabular}{|l|l|}
		\hline
		Task & \multicolumn{1}{c|}{Test Design} \\ \hline
		Integrity Level & 1 \\ \hline
		Methods & Evaluate the tryPLC() function and Browse Button\\ \hline
		Inputs &  PLC file selected via 'Browse' button \\ \hline
		Outputs &  'Success' notification displayed \\ \hline
		Expected Completion & After initialization, but before any trains dispatched\\ \hline
		Risks and Assumptions & \multicolumn{1}{p{4cm}|}{\raggedright PLC files exist in src/main/java/WaysideController/PLCResources (for automated testing, other external files may be used) }  \\ \hline
		Responsibility & Wayside Controller\\ \hline
	\end{tabular}
\end{table}

\subsection{Integration Tests}

\section{CTC Test Plan}
\subsection{Unit Tests}
\subsection{Integration Tests}

\section{MBO Test Plan}
\subsection{Unit Tests}
\subsection{Integration Tests}

\section{Train Model Test Plan}
\subsection{Unit Tests}
 
    \begin{table}[H]
        \centering
        \caption{Base Test A: Compute Velocity of train at rest with Power command of 100kW}
        \begin{tabular}{|l|l|}
            \hline
            Task & \multicolumn{1}{c|}{Test Design} \\ \hline
            Integrity Level & 2 \\ \hline
            Methods & Apply power command to train and compute velocity  \\ \hline
            Inputs &  Power Command input and "Start Test" button is pressed \\ \hline
            Outputs &  Velocity greater than 0 MPH will be produced  \\ \hline
            Expected Completion & \parbox[t]{10cm}{Test to be performed upon completion of complete submodule.\\ Expected date: March 24th}\\ \hline
            Risks and Assumptions & \parbox[t]{10cm}{The power command should be 100kW for this base case. \\ Assumption will be made that for base test train starts with 0 velocity} \\ \hline
            Responsibility & Train Model\\ \hline
        \end{tabular}
    \end{table}

	\begin{table}[H]
		\centering
		\caption{Repeat Base Test A with Power command greater than 100,000W}
		\begin{tabular}{|l|l|}
			\hline
			Task & \multicolumn{1}{c|}{Test Design} \\ \hline
			Integrity Level & 2 \\ \hline
			Methods & Apply higher power command to train and compute velocity  \\ \hline
			Inputs &  Power Command input and "Start Test" button is pressed \\ \hline
			Outputs &  Velocity greater than base case A will be produced  \\ \hline
			Expected Completion & \parbox[t]{10cm}{Test to be performed upon completion of complete submodule.\\ Expected date: March 24th}\\ \hline
			Risks and Assumptions & The power command should be a positive value greater than 100kW \\ \hline
			Responsibility & Train Model\\ \hline
		\end{tabular}
	\end{table}
    
    \begin{table}[H]
    	\centering
    	\caption{Repeat base test A with Power command less than 100,000W}
    	\begin{tabular}{|l|l|}
    		\hline
    		Task & \multicolumn{1}{c|}{Test Design} \\ \hline
    		Integrity Level & 2 \\ \hline
    		Methods & Apply lower power command to train and compute velocity  \\ \hline
    		Inputs &  Power Command input and "Start Test" button is pressed \\ \hline
    		Outputs &  Velocity lower than base case A will be produced  \\ \hline
    		Expected Completion & \parbox[t]{10cm}{Test to be performed upon completion of complete submodule.\\ Expected date: March 24th}\\ \hline
    		Risks and Assumptions & The power command should be a positive value smaller than 100kW \\ \hline
    		Responsibility & Train Model\\ \hline
    	\end{tabular}
    \end{table}

	\begin{table}[H]
		\centering
		\caption{Repeat base test A with grade of 3\%}
		\begin{tabular}{|l|l|}
			\hline
			Task & \multicolumn{1}{c|}{Test Design} \\ \hline
			Integrity Level & 2 \\ \hline
			Methods & Increase grade to 3\% and compute velocity  \\ \hline
			Inputs &  Grade set to 3\% and "Start Test" button is pressed \\ \hline
			Outputs &  Velocity lower than base case A will be produced  \\ \hline
			Expected Completion & \parbox[t]{10cm}{Test to be performed upon completion of complete submodule.\\ Expected date: March 24th}\\ \hline
			Risks and Assumptions & \parbox[t]{10cm}{The power command should be equal to 100kW \\and grade will be set to 3\% }\\ \hline
			Responsibility & Train Model\\ \hline
		\end{tabular}
	\end{table}

	\begin{table}[H]
		\centering
		\caption{Repeat base test A with grade of -3\%}
		\begin{tabular}{|l|l|}
			\hline
			Task & \multicolumn{1}{c|}{Test Design} \\ \hline
			Integrity Level & 2 \\ \hline
			Methods & Decrease grade to -3\% and compute velocity  \\ \hline
			Inputs &  Grade set to -3\% and "Start Test" button is pressed \\ \hline
			Outputs &  Velocity greater than base case A will be produced  \\ \hline
			Expected Completion & \parbox[t]{10cm}{Test to be performed upon completion of complete submodule.\\ Expected date: March 24th}\\ \hline
			Risks and Assumptions &\parbox[t]{10cm}{ The power command should be equal to 100kW \\and grade will be set to -3\%} \\ \hline
			Responsibility & Train Model\\ \hline
		\end{tabular}
	\end{table}
   
   	\begin{table}[H]
	   	\centering
	   	\caption{Repeat base test A with 150 passengers added}
	   	\begin{tabular}{|l|l|}
	   		\hline
	   		Task & \multicolumn{1}{c|}{Test Design} \\ \hline
	   		Integrity Level & 2 \\ \hline
	   		Methods & Increase passenger count to 150 and compute velocity  \\ \hline
	   		Inputs &  Number of passengers = 150 and "Start Test" button is pressed \\ \hline
	   		Outputs &  Velocity smaller than base case A will be produced  \\ \hline
	   		Expected Completion & \parbox[t]{10cm}{Test to be performed upon completion of complete submodule.\\ Expected date: March 24th}\\ \hline
	   		Risks and Assumptions & \parbox[t]{10cm}{The power command should be equal to 100k0W \\and 150 passengers will be added onboard the train} \\ \hline
	   		Responsibility & Train Model\\ \hline
	   	\end{tabular}
   \end{table}

	\begin{table}[H]
		\centering
		\caption{Base Case B: Compute Velocity of train at 25MPH with Power command of 100kW }
		\begin{tabular}{|l|l|}
			\hline
			Task & \multicolumn{1}{c|}{Test Design} \\ \hline
			Integrity Level & 2 \\ \hline
			Methods & Apply Power command of 100kW and compute new velocity  \\ \hline
			Inputs &  Power Command set to 100kW and "Start Test" button is pressed \\ \hline
			Outputs &  Velocity larger than 25MPH will be produced  \\ \hline
			Expected Completion & \parbox[t]{10cm}{Test to be performed upon completion of complete submodule.\\ Expected date: March 24th}\\ \hline
			Risks and Assumptions & The power command should be equal to 100kW \\ \hline
			Responsibility & Train Model\\ \hline
		\end{tabular}
	\end{table}

	\begin{table}[H]
		\centering
		\caption{Repeat Base Case B with power command of 0W }
		\begin{tabular}{|l|l|}
			\hline
			Task & \multicolumn{1}{c|}{Test Design} \\ \hline
			Integrity Level & 2 \\ \hline
			Methods & Apply Power command of 0W and compute new velocity  \\ \hline
			Inputs &  Power Command set to 0W and "Start Test" button is pressed \\ \hline
			Outputs &  Velocity smaller than 25MPH will be produced  \\ \hline
			Expected Completion & \parbox[t]{10cm}{Test to be performed upon completion of complete submodule.\\ Expected date: March 24th}\\ \hline
			Risks and Assumptions & The power command should be equal to 0W \\ \hline
			Responsibility & Train Model\\ \hline
		\end{tabular}
	\end{table}

	\begin{table}[H]
		\centering
		\caption{Repeat Base Case B with power command less than 0W }
		\begin{tabular}{|l|l|}
			\hline
			Task & \multicolumn{1}{c|}{Test Design} \\ \hline
			Integrity Level & 2 \\ \hline
			Methods & Apply negative power command and compute new velocity  \\ \hline
			Inputs &  Power Command set to -100kW and "Start Test" button is pressed \\ \hline
			Outputs &  Invalid input message will appear  \\ \hline
			Expected Completion & \parbox[t]{10cm}{Test to be performed upon completion of complete submodule.\\ Expected date: March 24th}\\ \hline
			Risks and Assumptions & Power command must be positive for all possible cases\\ \hline
		\end{tabular}
	\end{table}

	\begin{table}[H]
		\centering
		\caption{Repeat Base Case B with power command greater than 120kW }
		\begin{tabular}{|l|l|}
			\hline
			Task & \multicolumn{1}{c|}{Test Design} \\ \hline
			Integrity Level & 2 \\ \hline
			Methods & Apply power command above max and compute new velocity  \\ \hline
			Inputs &  Power Command set to 150kW and "Start Test" button is pressed \\ \hline
			Outputs &  Speed will remain 25MPH as there is no more power available  \\ \hline
			Expected Completion & \parbox[t]{10cm}{Test to be performed upon completion of complete submodule.\\ Expected date: March 24th}\\ \hline
			Risks and Assumptions & If power exceeds max, the velocity stays the same\\ \hline
		\end{tabular}
	\end{table}

	\begin{table}[H]
		\centering
		\caption{Repeat Base Case B but apply Service brakes}
		\begin{tabular}{|l|l|}
			\hline
			Task & \multicolumn{1}{c|}{Test Design} \\ \hline
			Integrity Level & 3 \\ \hline
			Methods & Apply service brakes and compute new velocity  \\ \hline
			Inputs &  Service brakes engaged and "Start Test" button is pressed \\ \hline
			Outputs & \parbox[t]{10cm}{ Service brake status switched to ON\\ Power Command set to 0\\ Train speed decreased to lower than 25 MPH  }\\ \hline
			Expected Completion & \parbox[t]{10cm}{Test to be performed upon completion of complete submodule.\\ Expected date: March 24th}\\ \hline
			Risks and Assumptions & Service brake will automatically override power command to 0W\\ \hline
		\end{tabular}
	\end{table}

	\begin{table}[H]
		\centering
		\caption{Repeat Base Case B but apply Emergency brakes}
		\begin{tabular}{|l|l|}
			\hline
			Task & \multicolumn{1}{c|}{Test Design} \\ \hline
			Integrity Level & 3 \\ \hline
			Methods & Apply Emergency brakes and compute new velocity  \\ \hline
			Inputs &  Emergency brakes engaged and "Start Test" button is pressed \\ \hline
			Outputs & \parbox[t]{10cm}{ Emergency brake status switched to ON\\ Power Command set to 0\\ Train speed decreased to lower than 25 MPH\\ Train speed also lower than service brake test case  }\\ \hline
			Expected Completion & \parbox[t]{10cm}{Test to be performed upon completion of complete submodule.\\ Expected date: March 24th}\\ \hline
			Risks and Assumptions & Service brake will automatically override power command to 0W\\ \hline
		\end{tabular}
	\end{table}

	\begin{table}[H]
		\centering
		\caption{Activate engine failure on moving train}
		\begin{tabular}{|l|l|}
			\hline
			Task & \multicolumn{1}{c|}{Test Design} \\ \hline
			Integrity Level & 2 \\ \hline
			Methods & Toggle engine failure status to on  \\ \hline
			Inputs &  Radio button for engine failure set to ON \\ \hline
			Outputs & \parbox[t]{10cm}{ Engine Failure status switched to ON\\ Power Command set to 0\\ Train speed decreased to 0 MPH\\ Service brake status set to ON  }\\ \hline
			Expected Completion & \parbox[t]{10cm}{Test to be performed upon completion of complete submodule.\\ Expected date: April 5th}\\ \hline
			Risks and Assumptions & Service brake will automatically activate on failure\\ \hline
		\end{tabular}
	\end{table}

	\begin{table}[H]
		\centering
		\caption{Activate Signal failure on moving train}
		\begin{tabular}{|l|l|}
			\hline
			Task & \multicolumn{1}{c|}{Test Design} \\ \hline
			Integrity Level & 2 \\ \hline
			Methods & Toggle signal failure status to on  \\ \hline
			Inputs &  Radio button for signal failure set to ON \\ \hline
			Outputs & \parbox[t]{10cm}{ Signal Failure status switched to ON\\ Power Command set to 0\\ Train speed decreased to 0 MPH\\ Service brake status set to ON  }\\ \hline
			Expected Completion & \parbox[t]{10cm}{Test to be performed upon completion of complete submodule.\\ Expected date: April 5th}\\ \hline
			Risks and Assumptions & Service brake will automatically activate on failure\\ \hline
		\end{tabular}
	\end{table}

	\begin{table}[H]
		\centering
		\caption{Activate Brake failure on moving train}
		\begin{tabular}{|l|l|}
			\hline
			Task & \multicolumn{1}{c|}{Test Design} \\ \hline
			Integrity Level & 2 \\ \hline
			Methods & Toggle brake failure status to on  \\ \hline
			Inputs &  Radio button for brake failure set to ON \\ \hline
			Outputs & \parbox[t]{10cm}{ Brake Failure status switched to ON\\ Power Command set to 0\\ Train speed decreased to 0 MPH\\ Emergency brake status set to ON  }\\ \hline
			Expected Completion & \parbox[t]{10cm}{Test to be performed upon completion of complete submodule.\\ Expected date: April 5th}\\ \hline
			Risks and Assumptions & Emergency brake will activate on failure in service brakes\\ \hline
		\end{tabular}
	\end{table}

	\begin{table}[H]
		\centering
		\caption{Open doors on moving train}
		\begin{tabular}{|l|l|}
			\hline
			Task & \multicolumn{1}{c|}{Test Design} \\ \hline
			Integrity Level & 2 \\ \hline
			Methods & Open left side doors on moving train  \\ \hline
			Inputs &  Radio button for left doors set to OPEN \\ \hline
			Outputs & Invalid action pop-up\\ \hline
			Expected Completion & \parbox[t]{10cm}{Test to be performed upon completion of complete submodule.\\ Expected date: April 5th}\\ \hline
			Risks and Assumptions & Doors will not open while train is in motion\\ \hline
		\end{tabular}
	\end{table}

	\begin{table}[H]
		\centering
		\caption{Open doors on non-moving train}
		\begin{tabular}{|l|l|}
			\hline
			Task & \multicolumn{1}{c|}{Test Design} \\ \hline
			Integrity Level & 2 \\ \hline
			Methods & Open left side doors on non-moving train  \\ \hline
			Inputs &  Radio button for left doors set to OPEN \\ \hline
			Outputs & Left door status on Train model changes to OPEN\\ \hline
			Expected Completion & \parbox[t]{10cm}{Test to be performed upon completion of complete submodule.\\ Expected date: April 5th}\\ \hline
			Risks and Assumptions &\parbox[t]{10cm}{ Left and Right doors can both be opened at the same time \\ but opening is independent}\\ \hline
		\end{tabular}
	\end{table}

	\begin{table}[H]
		\centering
		\caption{Power Command applied to non-moving train with open doors}
		\begin{tabular}{|l|l|}
			\hline
			Task & \multicolumn{1}{c|}{Test Design} \\ \hline
			Integrity Level & 2 \\ \hline
			Methods & Open left door, then apply power of 100kW  \\ \hline
			Inputs &  \parbox[t]{10cm}{Radio button for left doors set to OPEN\\ Power command Set to 100kW\\ "Start Test" button pressed }\\ \hline
			Outputs & Error message pop-up\\ \hline
			Expected Completion & \parbox[t]{10cm}{Test to be performed upon completion of complete submodule.\\ Expected date: April 5th}\\ \hline
			Risks and Assumptions & Train can not move if doors are open\\ \hline
		\end{tabular}
	\end{table}

	\begin{table}[H]
		\centering
		\caption{Power Command applied to non-moving train with Failure status}
		\begin{tabular}{|l|l|}
			\hline
			Task & \multicolumn{1}{c|}{Test Design} \\ \hline
			Integrity Level & 2 \\ \hline
			Methods & Engage any failure, then apply power of 100kW  \\ \hline
			Inputs &  \parbox[t]{10cm}{Radio button for Engine Failure set to ON\\ Power command Set to 100kW\\ "Start Test" button pressed }\\ \hline
			Outputs & Error message pop-up\\ \hline
			Expected Completion & \parbox[t]{10cm}{Test to be performed upon completion of complete submodule.\\ Expected date: April 5th}\\ \hline
			Risks and Assumptions & Train can not move if failures are present\\ \hline
		\end{tabular}
	\end{table}

	\begin{table}[H]
		\centering
		\caption{Power Command applied to non-moving train with engaged brakes}
		\begin{tabular}{|l|l|}
			\hline
			Task & \multicolumn{1}{c|}{Test Design} \\ \hline
			Integrity Level & 2 \\ \hline
			Methods & Engage either brake, then apply power of 100kW  \\ \hline
			Inputs &  \parbox[t]{10cm}{Service Brakes engaged\\ Power command Set to 100kW\\ "Start Test" button pressed }\\ \hline
			Outputs & Error message pop-up\\ \hline
			Expected Completion & \parbox[t]{10cm}{Test to be performed upon completion of complete submodule.\\ Expected date: April 5th}\\ \hline
			Risks and Assumptions & Train can not move if brakes are engaged\\ \hline
		\end{tabular}
	\end{table}

	\begin{table}[H]
		\centering
		\caption{Interior Light Test}
		\begin{tabular}{|l|l|}
			\hline
			Task & \multicolumn{1}{c|}{Test Design} \\ \hline
			Integrity Level & 1 \\ \hline
			Methods & Lights turned on in test console  \\ \hline
			Inputs &  \parbox[t]{10cm}{Radio button for lights set to ON\\ "Start Test" button pressed }\\ \hline
			Outputs &Interior Lights set to ON \\ \hline
			Expected Completion & \parbox[t]{10cm}{Test to be performed upon completion of complete submodule.\\ Expected date: April 5th}\\ \hline
			Risks and Assumptions & Lights can be on at any time.\\ \hline
		\end{tabular}
	\end{table}

	\begin{table}[H]
		\centering
		\caption{Train Temperature set to 60F Thermostat set to 65F}
		\begin{tabular}{|l|l|}
			\hline
			Task & \multicolumn{1}{c|}{Test Design} \\ \hline
			Integrity Level & 1 \\ \hline
			Methods & Set intial temp to 60F, Set Thermostat to 65F  \\ \hline
			Inputs &  \parbox[t]{10cm}{Train Temperature set to 60F\\ Thermostat set to 65F\\ "Start Test" button pressed }\\ \hline
			Outputs &\parbox[t]{10cm}{ Heater set to ON\\AC set to OFF\\ Temperature increases to 65F} \\ \hline
			Expected Completion & \parbox[t]{10cm}{Test to be performed upon completion of complete submodule.\\ Expected date: April 5th}\\ \hline
			Risks and Assumptions & Heater and AC can not be on at the same time.\\ \hline
		\end{tabular}
	\end{table}



	\begin{table}[H]
		\centering
		\caption{Train Temperature set to 60F Thermostat set to 60F}
		\begin{tabular}{|l|l|}
			\hline
			Task & \multicolumn{1}{c|}{Test Design} \\ \hline
			Integrity Level & 1 \\ \hline
			Methods & Set intial temp to 60F, Set Thermostat to 60F  \\ \hline
			Inputs &  \parbox[t]{10cm}{Train Temperature set to 60F\\ Thermostat set to 60F\\ "Start Test" button pressed }\\ \hline
			Outputs &\parbox[t]{10cm}{ Heater set to OFF\\AC set to OFF\\ Temperature does not change} \\ \hline
			Expected Completion & \parbox[t]{10cm}{Test to be performed upon completion of complete submodule.\\ Expected date: April 5th}\\ \hline
			Risks and Assumptions & \parbox[t]{10cm}{Temperature can only change if heat or AC is on\\ No heat loss due to windows open}\\ \hline
		\end{tabular}
	\end{table}

	\begin{table}[H]
		\centering
		\caption{Train Temperature set to 60F Thermostat set to 55F}
		\begin{tabular}{|l|l|}
			\hline
			Task & \multicolumn{1}{c|}{Test Design} \\ \hline
			Integrity Level & 1 \\ \hline
			Methods & Set intial temp to 60F, Set Thermostat to 55F  \\ \hline
			Inputs &  \parbox[t]{10cm}{Train Temperature set to 60F\\ Thermostat set to 55F\\ "Start Test" button pressed }\\ \hline
			Outputs &\parbox[t]{10cm}{ Heater set to OFF\\AC set to ON\\ Temperature decreases to 55F} \\ \hline
			Expected Completion & \parbox[t]{10cm}{Test to be performed upon completion of complete submodule.\\ Expected date: April 5th}\\ \hline
			Risks and Assumptions & Heater and AC can not be on at the same time.\\ \hline
		\end{tabular}
	\end{table}

	\begin{table}[H]
		\centering
		\caption{Integration test With train Controller}
		\begin{tabular}{|l|l|}
			\hline
			Task & \multicolumn{1}{c|}{Test Design} \\ \hline
			Integrity Level & 1 \\ \hline
			Methods & Repeat above tests but receiving values from Train Controller \\ \hline
			Inputs &  Power Command, Utility statuses, Brake Statuses\\ \hline
			Outputs & Outputs should reflect similar results as the test cases above \\ \hline
			Expected Completion & \parbox[t]{10cm}{Test to be performed upon completion of integration with train controller.\\ Expected date: April 7th}\\ \hline
			Risks and Assumptions & \parbox[t]{10cm}{Whether inputs come from train controller or test console results should be the same}\\ \hline
		\end{tabular}
	\end{table}

	\begin{table}[H]
		\centering
		\caption{Integration test With Track Model}
		\begin{tabular}{|l|l|}
			\hline
			Task & \multicolumn{1}{c|}{Test Design} \\ \hline
			Integrity Level & 1 \\ \hline
			Methods & Read in values for current block's grade for calculations \\ \hline
			Inputs & Request for current grade to track Model\\ \hline
			Outputs & If successful a grade will be returned to train model \\ \hline
			Expected Completion & \parbox[t]{10cm}{Test to be performed upon completion of integration with Track Model.\\ Expected date: April 7th}\\ \hline
			Risks and Assumptions & Track model will send grade upon entrance to block\\ \hline
		\end{tabular}
	\end{table}

	\begin{table}[H]
		\centering
		\caption{Integration test With MBO}
		\begin{tabular}{|l|l|}
			\hline
			Task & \multicolumn{1}{c|}{Test Design} \\ \hline
			Integrity Level & 1 \\ \hline
			Methods & Testing communication between MBo and Train model \\ \hline
			Inputs & Request for current location from MBO\\ \hline
			Outputs & If successful a location will be sent to the MBO \\ \hline
			Expected Completion & \parbox[t]{10cm}{Test to be performed upon completion of integration with MBO.\\ Expected date: April 7th}\\ \hline
			Risks and Assumptions & Train model will periodically update MBO with location\\ \hline
		\end{tabular}
	\end{table}
\subsection{Integration Tests}

\section{Train Controller Test Plan}
\subsection{Unit Tests}
\subsection{Integration Tests}

\section{Changelog}

\begin{table}[H]
	\centering
	\caption{Change}
	\label{changelogl}
	\begin{tabular}{|l|l|}
		\hline
		Date & \multicolumn{1}{c|}{Change} \\ \hline
		March 14, 2017 & General Test Plan \\ \hline
	\end{tabular}
\end{table}
\end{document}          
