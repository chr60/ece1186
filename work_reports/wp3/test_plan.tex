\documentclass[]{article}
\usepackage{hyperref}
\usepackage{multirow}
\usepackage{float}
% Title Page
\title{Low-Comotovation: Group Test Plan}
\author{Michael Ghaben}
\date{}

\begin{document}
\maketitle
\tableofcontents
\titlepage
\section{Introduction}
\subsection{Document Identifier}
This document is the Low-Comotovation Group Test Plan for the design review phase of the Spring 2017 software engineering project.. In this document we detail the testing specifications, requirements, and procedures for the implementation and evaluation of testing procedures.

\subsection{Scope}
In this document a number of assumptions regarding the projects scope and lifecycle are made. Specifically:

\begin{enumerate}
	\item Due to the short development cycle associated with this project, some non-negligble defects will likely persist
	\item Under the iterative development methodology this group is following, the development of additional tests may be completed to address defects is expected
\end{enumerate}
To better address these constraints, we will utilize a continuous integration methodology around iterative development and testing.So that we may ensure rapid development with minimal time overhead, we shall utilize continuous integration tools and methodology. Testing will be broadly divided into two categories: subsystem, and integration. Subsystem testing will be primarily focused on testing an individual system to ensure minimal functionality of a given subsystem and be done primarily independently by each group member. Integration testing will be designated as tests which require two or more modules functionality to satisfy requirements.

Specifically, it is anticipated that software defects will be found and require that testing which was not anticipated. As a result, we shall primarily focus on defining general tsts to be implemented and leave the exact testing requirements as implementation is completed. Additionally, will primarily focus on subsytems to be delivered to the end-user, the train company.

Examples of these subsystems would be the CTC module or the train controller, which are expected to be integrated into the final deliverable. Subsystems which are not examples of the final deliverables are the Train Model and Track Model subsystems, as they will ultimately be removed and replaced with the physical subsystems.

\subsection{System Overview and Key Features}
The purpose of the system under development is a to provide a train system for the Pittsburgh North Shore Rail system.

This system broadly consists of 6 subsystems:
\begin{enumerate}
	\item  Track Model - a physical model of a track to be used for testing
	\item Train Model - a physical model of a train to be used for testing
	\item CTC System - A CTC system to be implemented on the final train system
	\item MBO - Schedules trains and implements Fixed Block Mode and Moving Block Overlay mode
	\item Wayside System - A wayside for coordinating with all models
\end{enumerate}
For a further discussion of the system, we refer the reader to the project reuqirements and discussion board.

\section{Test Overview}
The test organization is broadly divided into two sections: subsystem and integration testing. Subsystem tests are regarded as tests associated with only a single system at a time. Integration tests broadly refer to the tests of the integration of more than a single subsystem. In this view, testing is accomplished by each subsystem independently at the discresion of the individual developing the subsystem. Then, as members of the team develop their subsystem, each member shall attempt to integrate and develop tests for their integration. Due to the nature of continuous integration, tests are expected to be developed in parallel to the development of the program modules. 

\subsection{Master Test Schedule}
The test schedul will be implemented as follows::

\subsection{Integrity Level Schema}
For this project, we utilize three integrity levels. The integrity levels for this project are as follows:
\begin{enumerate}
	\item The lowest severity level. This is reserved for tasks which will ultimately not be passed onto the finished product and pose no threat to catastrophic failure. 
	\item This severity level is reserved for failures which may lead to errors in subsystem integration or incorrect information delivered to critical subsystems
	\item The most severe integrity level. Is a vital system or otherwise threatens life or limb in an imlemented subsystem
\end{enumerate}
\subsection{Responsibilities}
Michael Ghaben will be responsible for the integrity of the automated build system as well as the integrity of the master branch.

\subsection{Tools, Techniques, and Metrics}
To implement the test environment and better facilitate a continuous integration test environment, we utilize Travis-CI\footnote{Travis-CI.org} with GitHub\footnote{github.com} integration with Slack\footnote{slack.com} and Jupiter JUnit \footnote{junit.org} integration. By utllizing these tools, we allow automated tests to be run remotely using the Maven\footnote{maven.apache.org}. This allows for rapid feedback into the developmental process, allowing for better integration and testing of the team. The usage of these tools creates feedback loop between implementation, testing, and integration, leading to significant productivity gains. In each case ,testing will be carried out remotely by the build system.

The system shall be quantitatively evaluated on percentage of percentage of test passing. Each subsystem shall be responsible for the determination of importance of testing individual components of their subsystem, with the exception of vital components.

\section{Details}
\subsection{Process}
A test shall be detailed in the following manner:
\begin{table}[H]
	\centering
	\caption{Test Plan}
	\label{my-label}
	\begin{tabular}{|l|l|}
		\hline
		Task & \multicolumn{1}{c|}{Test Design} \\ \hline
		Integrity Level & What is the integrity level? \\hline
		Methods & How? \\ \hline
		Inputs &  What Inputs? \\ \hline
		Outputs &  What is a successful output?\\ \hline
		Expected Completion & When will it be done?\\ \hline
		Risks and Assumptions & What are you assuming?\\ \hline
		Responsibility & Who are you?\\ \hline
	\end{tabular}
\end{table}
In this, we expect to utilize both integration as well as unit tests. After each member pushes to GitHub on his or her respective branch, Travis-CI will provide regression testing on all unit and integration tests that have been implemented. To merge into master, all tests must be passing. 

\subsection{Test Documentation Requirements}
Each test shall be documented using the above table as well as any auxillary information by whomever holds testing responsibilities. Each module shall have a specified test plan for each module covering their individual component for testing. Each subsystem test plan shall detail unit tests for his or her own module. Additionally, integration testing will be accomplished in a similar fashon completed by the group.

Furthermore, the unit and integration testing procedure will be supplemented by functional testing. Each group member shall conduct user testing of each other module. During this time, the testing member will attempt to cause defective behavior at any level. These defects will be tracked via GitHub issues. This testing will occur weekly.

\subsection{Test Administration Requirements}
For a unit or integration test to be considered complete, it must successfully build on the build server utilizing the Maven build system. This ensures consistent repeatable builds to attempt to ensure the clients functional requirements will be met.

Additionally, for functional testing it is expected that each group member submits either a bug report via GitHub issues. Should a group member fail to find any defects and register them, he or she must challenge Professor Profeta to find a defect at the next class meeting. If the professor discovers a defect, the group member(s) who failed to find defects owe the other group members pizza. It is hoped that this procedure will lead to people finding more defects.
\subsection{Test Reporting Requirements}
Each written unit and integrationtest report will be provided by the Maven automated build system. 

To document user testing (e.g. by working with the user interface and attempting to find defects in that manner), a developmer may supplement this test report with the following syntax to automatically document the bug utilizng the following syntax:\newline \newline
/** \newline
* @bug $<Descriptive$ $ message>$ \newline
*/\newline
\newline
This will lead to documentation of the defect in the autmatically generated Doxygen documentation. Note that this should be used for defects which  are not necessarily covered under tests at a given time. By utilizing this process, an iterative cycle of development these defects may be tracked and inclusion in later tests to ensure proper functionality.

\section{Track Module Test Plan}
Author: Michael Ghaben

\subsection{Unit Tests}
\begin{table}[H]
	\centering
	\caption{CSV Reading Test Plan}
	\begin{tabular}{|l|l|}
		\hline
		Task & \multicolumn{1}{c|}{Test Design} \\ \hline
		Integrity Level & 1 \\ \hline
		Methods & Evaluate the readCSV function\\ \hline
		Inputs &  The files redline.csv \\ \hline
		Outputs &  The track model successfully reading the redline csv files\\ \hline
		Expected Completion & March 20, 2017\\ \hline
		Risks and Assumptions & Both redline and greenline have been properly input to the files\\ \hline
		Responsibility & Track Model\\ \hline
		\\ \hline
		Tested By   &  Michael Ghaben\\	\hline
		Date Tested & \parbox[t]{10cm}{April 14th}\\ \hline
		Results & Success\\ \hline
	\end{tabular}
\end{table}



\begin{table}[H]
	\centering
	\caption{CSV Reading Test Plan}
	\begin{tabular}{|l|l|}
		\hline
		Task & \multicolumn{1}{c|}{Test Design} \\ \hline
		Integrity Level & 1 \\ \hline
		Methods & Evaluate the readCSV function on green line\\ \hline
		Inputs &  The files greenline.csv \\ \hline
		Outputs &  The track model successfully reading the redline csv files\\ \hline
		Expected Completion & March 15, 2017\\ \hline
		Risks and Assumptions & \parbox[t]{10cm}{Both redline and greenline have been properly input to the csv files}\\ \hline
		Responsibility & Track Model\\ \hline
		\\ \hline
		Tested By   &  Michael Ghaben\\	\hline
		Date Tested & \parbox[t]{10cm}{April 19th}\\ \hline
		Results & Partial Success: Unit test passing but integration failing\\ \hline
	\end{tabular}
\end{table}

\begin{table}[H]
	\centering
	\caption{Test Switch Root Node Reading}
	\begin{tabular}{|l|l|}
		\hline
		Task & \multicolumn{1}{c|}{Test Design} \\ \hline
		Integrity Level & 2 \\ \hline
		Methods & \parbox[t]{10cm}{Evaluate the switch nodes association in TrackModel}\\ \hline
		Inputs &  The file redline.csv \\ \hline
		Outputs &  The track model successfully holding the root nodes in the rootMap\\ \hline
		Expected Completion & March 15, 2017\\ \hline
		Risks and Assumptions & Both redline and greenline have been properly input to the csv files \\ \hline
		Responsibility & Track Model\\ \hline
		\\ \hline
		Tested By   &  Michael Ghaben\\	\hline
		Date Tested & \parbox[t]{10cm}{April 14th}\\ \hline
		Results &Success\\ \hline

	\end{tabular}
\end{table}

\begin{table}[H]
	\centering
	\caption{Test Switch Root Node Reading}
	\begin{tabular}{|l|l|}
		\hline
		Task & \multicolumn{1}{c|}{Test Design} \\ \hline
		Integrity Level & 2 \\ \hline
		Methods & Evaluate the switch nodes association in TrackModel\\ \hline
		Inputs &  The file greenline.csv \\ \hline
		Outputs &  The track model successfully holding the root nodes in the rootMap\\ \hline
		Expected Completion & March 15, 2017\\ \hline
		Risks and Assumptions & Both redline and greenline have been properly input to the csv files \\ \hline
		Responsibility & Track Model\\ \hline
		\\ \hline
		Tested By   &  Michael Ghaben\\	\hline
		Date Tested & \parbox[t]{10cm}{April 15th}\\ \hline
		Results & Success\\ \hline

	\end{tabular}
\end{table}

\begin{table}[H]
	\centering
	\caption{Test Switch Node Leaf Reading}
	\begin{tabular}{|l|l|}
		\hline
		Task & \multicolumn{1}{c|}{Test Design} \\ \hline
		Integrity Level & 2 \\ \hline
		Methods & Evaluate the leaf nodes association in TrackModel\\ \hline
		Inputs &  The file redline.csv \\ \hline
		Outputs & \parbox[t]{10cm}{ The track model successfully holding the proper references in the Block object set by rootMap}\\ \hline
		Expected Completion & March 15, 2017\\ \hline
		Risks and Assumptions & Both redline and greenline have been properly input to the csv files \\ \hline
		Responsibility & Track Model\\ \hline
		\\ \hline
		Tested By   &  Michael Ghaben\\	\hline
		Date Tested & \parbox[t]{10cm}{April 14th}\\ \hline
		Results &Success\\ \hline
	\end{tabular}
\end{table}

\begin{table}[H]
	\centering
	\caption{Test Switch Node Leaf Reading}
	\begin{tabular}{|l|l|}
		\hline
		Task & \multicolumn{1}{c|}{Test Design} \\ \hline
		Integrity Level & 2 \\ \hline
		Methods & Evaluate the leaf nodes association in TrackModel\\ \hline
		Inputs &  The file greenline.csv \\ \hline
		Outputs &  \parbox[t]{10cm}{The track model successfully holding the proper refererences to the leaf nodes in the Block object set by rootMap}\\ \hline
		Expected Completion & March 15, 2017\\ \hline
		Risks and Assumptions & Both redline and greenline have been properly input to the csv files \\ \hline
		Responsibility & Track Model\\ \hline
		\\ \hline
		Tested By   &  Michael Ghaben\\	\hline
		Date Tested & \parbox[t]{10cm}{April 19th}\\ \hline
		Results & \parbox[t]{10cm}{Manual testing on switches were successful but linkage and integration is non-functional}\\ \hline
	\end{tabular}
\end{table}

\begin{table}[H]
	\centering
	\caption{Test Switch Node Leaf Reading}
	\begin{tabular}{|l|l|}
		\hline
		Task & \multicolumn{1}{c|}{Test Design} \\ \hline
		Integrity Level & 2 \\ \hline
		Methods & Evaluate the switching functionality in the red line\\ \hline
		Inputs &  The file redline.csv \\ \hline
		Outputs &  The proper  block given a non-default switch state\\ \hline
		Expected Completion & March 15, 2017\\ \hline
		Risks and Assumptions & Both redline and greenline have been properly input to the csv files \\ \hline
		Responsibility & Track Model\\ \hline
		\\ \hline
		Tested By   &  Michael Ghaben\\	\hline
		Date Tested & \parbox[t]{10cm}{April 19th}\\ \hline
		Results & \parbox[t]{10cm}{Manual testing on switches were successful but linkage and integration is non-functional}\\ \hline
	\end{tabular}
\end{table}

\begin{table}[H]
	\centering
	\caption{Test nextBlockForward() Red Line}
	\begin{tabular}{|l|l|}
		\hline
		Task & \multicolumn{1}{c|}{Test Design} \\ \hline
		Integrity Level & 2 \\ \hline
		Methods & Evaluate the functionality of the nextBlockForward() function on the redline \\ \hline
		Inputs &  The file redline.csv \\ \hline
		Outputs &  The proper block given a switch on the red line\\ \hline
		Expected Completion & March 15, 2017\\ \hline
		Risks and Assumptions & Both redline and greenline have been properly input to the csv files \\ \hline
		Responsibility & Track Model\\ \hline
		\\ \hline
		Tested By   &  Michael Ghaben\\	\hline
		Date Tested & \parbox[t]{10cm}{April 16th}\\ \hline
		Results & Success\\ \hline
	\end{tabular}
\end{table}

\begin{table}[H]
	\centering
	\caption{Test nextBlockForward() Green Line}
	\begin{tabular}{|l|l|}
		\hline
		Task & \multicolumn{1}{c|}{Test Design} \\ \hline
		Integrity Level & 2 \\ \hline
		Methods & Evaluate the functionality of the nextBlockForward() function on the green line \\ \hline
		Inputs &  The file greenline.csv \\ \hline
		Outputs &  The proper block given a switch on the green line\\ \hline
		Expected Completion & March 15, 2017\\ \hline
		Risks and Assumptions & Both redline and greenline have been properly input to the csv files \\ \hline
		Responsibility & Track Model\\ \hline
		\\ \hline
		Tested By   &  Michael Ghaben\\	\hline
		Date Tested & \parbox[t]{10cm}{April 19th}\\ \hline
		Results & \parbox[t]{10cm}{Unit testing is successful but integration of green line is non-functional}\\ \hline
	\end{tabular}
\end{table}

\begin{table}[H]
	\centering
	\caption{Test nextBlockBackward() Red Line}
	\begin{tabular}{|l|l|}
		\hline
		Task & \multicolumn{1}{c|}{Test Design} \\ \hline
		Integrity Level & 2 \\ \hline
		Methods & \parbox[t]{10cm}{Evaluate the functionality of the nextBlockBackward() function on the red line} \\ \hline
		Inputs &  The file redline.csv \\ \hline
		Outputs &  The proper block given a switch on the red line\\ \hline
		Expected Completion & March 15, 2017\\ \hline
		Risks and Assumptions & Both redline and greenline have been properly input to the csv files \\ \hline
		Responsibility & Track Model\\ \hline
		\\ \hline
		Tested By   &  Michael Ghaben\\	\hline
		Date Tested & \parbox[t]{10cm}{April 14th}\\ \hline
		Results & Success\\ \hline
	\end{tabular}
\end{table}

\begin{table}[H]
	\centering
	\caption{Test nextBlockBackward() Green Line}
	\begin{tabular}{|l|l|}
		\hline
		Task & \multicolumn{1}{c|}{Test Design} \\ \hline
		Integrity Level & 2 \\ \hline
		Methods & \parbox[t]{10cm}{Evaluate the functionality of the nextBlockBackward() function on the red line} \\ \hline
		Inputs &  The file redline.csv \\ \hline
		Outputs &  The proper block given a switch on the red line\\ \hline
		Expected Completion & March 15, 2017\\ \hline
		Risks and Assumptions & Both redline and greenline have been properly input to the csv files \\ \hline
		Responsibility & Track Model\\ \hline
		\\ \hline
		Tested By   &  Michael Ghaben\\	\hline
		Date Tested & \parbox[t]{10cm}{April 19th}\\ \hline
		Results &\parbox[t]{10cm}{ Unit testing is successful but integration of green line is non-functional}\\ \hline
	\end{tabular}
\end{table}

\begin{table}[H]
	\centering
	\caption{Test nextBlockBackward() Secondary Switch Conditions Red Line}
	\begin{tabular}{|l|l|}
		\hline
		Task & \multicolumn{1}{c|}{Test Design} \\ \hline
		Integrity Level & 2 \\ \hline
		Methods & \parbox[t]{10cm}{Evaluate the functionality of the nextBlockBackward() function on the red line under the alternate switch functionality}\\ \hline
		Inputs &  The file redline.csv \\ \hline
		Outputs &  The proper block given a switch on the red line\\ \hline
		Expected Completion & March 15, 2017\\ \hline
		Risks and Assumptions & Both redline and greenline have been properly input to the csv files \\ \hline
		Responsibility & Track Model\\ \hline
		\\ \hline
		Tested By   &  Michael Ghaben\\	\hline
		Date Tested & \parbox[t]{10cm}{April 19th}\\ \hline
		Results & Success\\ \hline
	\end{tabular}
\end{table}

\begin{table}[H]
	\centering
	\caption{Test nextBlockBackward() Secondary Switch Conditions Green Line}
	\begin{tabular}{|l|l|}
		\hline
		Task & \multicolumn{1}{c|}{Test Design} \\ \hline
		Integrity Level & 2 \\ \hline
		Methods & \parbox[t]{10cm}{Evaluate the functionality of the nextBlockBackward() function on the red line under the alternate switch functionality} \\ \hline
		Inputs &  The file greenline.csv \\ \hline
		Outputs &  The proper block given a switch on the red line\\ \hline
		Expected Completion & Before Half-Life 3\\ \hline
		Risks and Assumptions & Both redline and greenline have been properly input to the csv files \\ \hline
		Responsibility & Track Model\\ \hline
		\\ \hline
		Tested By   &  Michael Ghaben\\	\hline
		Date Tested & \parbox[t]{10cm}{April 19th}\\ \hline
		Results & \parbox[t]{10cm}{Unit testing is successful but integration of green line is non-functional}\\ \hline
	\end{tabular}
\end{table}

\begin{table}[H]
	\centering
	\caption{Test Station Arrival/Departure Time}
	\begin{tabular}{|l|l|}
		\hline
		Task & \multicolumn{1}{c|}{Test Station Arrival and Departure Time setting} \\ \hline
		Integrity Level & 1 \\ \hline
		Methods & Evaluate the functionality of the ability to set arrival and departure times at a given station \\ \hline
		Inputs &  Arrival and departure time \\ \hline
		Outputs &  \parbox[t]{10cm}{The proper time set in a station}\\ \hline
		Expected Completion & April 1, 2017\\ \hline
		Risks and Assumptions & That the test will not interact with other functionality \\ \hline
		Responsibility & Track Model\\ \hline
		\\ \hline
		Tested By   &  Michael Ghaben\\	\hline
		Date Tested & \parbox[t]{10cm}{April 19th}\\ \hline
		Results & \parbox[t]{10cm}{Unit test success but no station arrival and departure time is available from MBO}\\ \hline
	\end{tabular}
\end{table}

\begin{table}[H]
	\centering
	\caption{Test Station Passenger Loading}
	\begin{tabular}{|l|l|}
		\hline
		Task & \multicolumn{1}{c|}{Validate the usage of loading passengers from station to train} \\ \hline
		Integrity Level & 1 \\ \hline
		Methods & Evaluate the functionality of the ability to load passengers for multiple input values \\ \hline
		Inputs &  Maximum number of passengers\\ \hline
		Outputs &  \parbox[t]{10cm}{Number of passengers to be added}\\ \hline
		Expected Completion & April 1, 2017\\ \hline
		Risks and Assumptions & The input will be an Integer type \\ \hline
		Responsibility & Track Model\\ \hline
		\\ \hline
		Tested By   &  Michael Ghaben\\	\hline
		Date Tested & \parbox[t]{10cm}{April 19th}\\ \hline
		Results & Success\\ \hline
	\end{tabular}
\end{table}

\begin{table}[H]
	\centering
	\caption{Test Station Passenger Unloading}
	\begin{tabular}{|l|l|}
		\hline
		Task & \multicolumn{1}{c|}{Validate the usage of unloading passengers from train to station} \\ \hline
		Integrity Level & 1 \\ \hline
		Methods & Evaluate the functionality of the ability to unload passengers for multiple input values \\ \hline
		Inputs &  Number of passengers unloaded\\ \hline
		Outputs &  \parbox[t]{10cm}{None}\\ \hline
		Expected Completion & April 1, 2017\\ \hline
		Risks and Assumptions & The input will be an Integer type \\ \hline
		Responsibility & Track Model\\ \hline
		\\ \hline
		Tested By   &  Michael Ghaben\\	\hline
		Date Tested & \parbox[t]{10cm}{April 19th}\\ \hline
		Results & Success\\ \hline

	\end{tabular}
\end{table}




\subsection{Integration Tests}

\begin{table}[H]
	\centering
	\caption{Test Track Controller Switching}
	\begin{tabular}{|l|l|}
		\hline
		Task & \multicolumn{1}{c|}{Test Design} \\ \hline
		Integrity Level & 2 \\ \hline
		Methods & Evaluate the functionality of the track controller to switch a switch state \\ \hline
		Inputs &  The file redline.csv \\ \hline
		Outputs &  \parbox[t]{10cm}{The proper block given a switch on the red line and the MBO switching the switch}\\ \hline
		Expected Completion & April 15, 2017\\ \hline
		Risks and Assumptions & Both redline and greenline switches are able to be toggled successfully \\ \hline
		Responsibility & Track Model\\ \hline
		\\ \hline
		Tested By   &  Michael Ghaben\\	\hline
		Date Tested & \parbox[t]{10cm}{April 19th}\\ \hline
		Results & Success\\ \hline

	\end{tabular}
\end{table}

\begin{table}[H]
	\centering
	\caption{Test Track Controller Switching}
	\begin{tabular}{|l|l|}
		\hline
		Task & \multicolumn{1}{c|}{Test Design} \\ \hline
		Integrity Level & 2 \\ \hline
		Methods & Evaluate the functionality of the track controller to switch a switch state \\ \hline
		Inputs &  The file greenline.csv \\ \hline
		Outputs &  \parbox[t]{10cm}{The proper block given a switch on the green line and the MBO switching the switch}\\ \hline
		Expected Completion & April 1, 2017\\ \hline
		Risks and Assumptions & Both redline and greenline switches are able to be toggled successfully \\ \hline
		Responsibility & Track Model\\ \hline
		\\ \hline
		Tested By   &  Michael Ghaben\\	\hline
		Date Tested & \parbox[t]{10cm}{April 19th}\\ \hline
		Results & \parbox[t]{10cm}{Unit testing is successful but integration of green line is non-functional}\\ \hline
	\end{tabular}
\end{table}

\begin{table}[H]
	\centering
	\caption{Test Train Door Side}
	\begin{tabular}{|l|l|}
		\hline
		Task & \multicolumn{1}{c|}{Test Design} \\ \hline
		Integrity Level & 1 \\ \hline
		Methods &  \parbox[t]{10cm}{Evaluate the functionality of the ability to relay the proper door side of train to open based upon incoming direction of approach to a station} \\ \hline
		Inputs &  The beacon info called by the train controller \\ \hline
		Outputs &  \parbox[t]{10cm}{The proper approach side}\\ \hline
		Expected Completion & April 15, 2017\\ \hline
		Risks and Assumptions & \parbox[t]{10cm}{That the communication between train model and train controller will be successful} \\ \hline
		Responsibility & Track Model, Train Model and Train Controller\\ \hline
		\\ \hline
		Tested By   &  Michael Ghaben\\	\hline
		Date Tested & \parbox[t]{10cm}{April 19th}\\ \hline
		Results & Success\\ \hline
	\end{tabular}
\end{table}

\begin{table}[H]
	\centering
	\caption{Test Setting Speed and Authority}
	\begin{tabular}{|l|l|}
		\hline
		Task & \multicolumn{1}{c|}{Validate the capability of the Track Controller to set speed and authority} \\ \hline
		Integrity Level & 2\\ \hline
		Methods & Evaluate the functionality of the ability to set speed and authority \\ \hline
		Inputs &  Set speed and authority\\ \hline
		Outputs &  \parbox[t]{10cm}{None}\\ \hline
		Expected Completion & April 15, 2017\\ \hline
		Risks and Assumptions & The input will be a valid speed and authority \\ \hline
		Responsibility & Track Model and Track Controller\\ \hline
		\\ \hline
		Tested By   &  Michael Ghaben\\	\hline
		Date Tested & \parbox[t]{10cm}{April 19th}\\ \hline
		Results & Success\\ \hline
	\end{tabular}
\end{table}

\section{Track Controller Test Plan}
\subsection{Unit Tests}
%UNIT TEST
\begin{table}[H]
	\centering
	\caption{Load PLC Program via Browse Button}
	\begin{tabular}{|l|l|}
		\hline
		Task & \multicolumn{1}{c|}{Test Design} \\ \hline
		Integrity Level & 1 \\ \hline
		Methods & Evaluate the tryPLC() function and Browse Button\\ \hline
		Inputs &  PLC file selected via 'Browse' button, then clicking 'Load' button \\ \hline
		Outputs &  'Success' notification displayed \\ \hline
		Expected Completion & After initialization, but before any trains dispatched\\ \hline
		Risks and Assumptions & \parbox[t]{10cm}{For automated testing, PLC files exist in PLCResources Folder \\(Other external files may be used normally)}  \\ \hline
		Responsibility & Wayside Controller\\ \hline
		\\ \hline
		Tested By   &  Max Reno\\	\hline
		Date Tested & \parbox[t]{10cm}{April 19th}\\ \hline
		Results & FILL IN YOUR RESULTS HERE (SUCCESS/FAIL/REASON(If fail))\\ \hline
	\end{tabular}
\end{table}

%UNIT TEST
\begin{table}[H]
	\centering
	\caption{Load PLC Program via File Path}
	\begin{tabular}{|l|l|}
		\hline
		Task & \multicolumn{1}{c|}{Test Design} \\ \hline
		Integrity Level & 1 \\ \hline
		Methods & Evaluate the tryPLC() function\\ \hline
		Inputs &  PLC file selected via entering file path, then clicking 'Load' button \\ \hline
		Outputs &  'Success' notification displayed \\ \hline
		Expected Completion & After initialization, but before any trains dispatched\\ \hline
		Risks and Assumptions & \parbox[t]{10cm}{For automated testing, PLC files exist in PLCResources Folder \\(Other external files may be used normally)} \\ \hline
		Responsibility & Wayside Controller\\ \hline
		\\ \hline
		Tested By   &  Max Reno\\	\hline
		Date Tested & \parbox[t]{10cm}{April 19th}\\ \hline
		Results & FILL IN YOUR RESULTS HERE (SUCCESS/FAIL/REASON(If fail))\\ \hline
	\end{tabular}
\end{table}

%UNIT TEST
\begin{table}[H]
	\centering
	\caption{PLC Logic Calculation}
	\begin{tabular}{|l|l|}
		\hline
		Task & \multicolumn{1}{c|}{Test Design} \\ \hline
		Integrity Level & 2 \\ \hline
		Methods & Evaluate runPLC() function\\ \hline
		Inputs &  Current block \\ \hline
		Outputs &  Boolean [] of necessary track state \\ \hline
		Expected Completion & After calculation before next set of data is passed\\ \hline
		Risks and Assumptions & \parbox[t]{10cm}{Valid block is passed to function} \\ \hline
		Responsibility & Wayside Controller\\ \hline
		\\ \hline
		Tested By   &  Max Reno\\	\hline
		Date Tested & \parbox[t]{10cm}{April 19th}\\ \hline
		Results & FILL IN YOUR RESULTS HERE (SUCCESS/FAIL/REASON(If fail))\\ \hline
	\end{tabular}
\end{table}

%UNIT TEST
\begin{table}[H]
	\centering
	\caption{View Block info}
	\begin{tabular}{|l|l|}
		\hline
		Task & \multicolumn{1}{c|}{Test Design} \\ \hline
		Integrity Level & 1 \\ \hline
		Methods & Tests dropdown menu functionality\\ \hline
		Inputs &  Select of Line, Section, Block \\ \hline
		Outputs &  Block\\ \hline
		Expected Completion & As soon as user makes selection, info is returned\\ \hline
		Risks and Assumptions & \parbox[t]{10cm}{Current information returned is updated} \\ \hline
		Responsibility & Wayside Controller\\ \hline
		\\ \hline
		Tested By   &  Max Reno\\	\hline
		Date Tested & \parbox[t]{10cm}{April 19th}\\ \hline
		Results & FILL IN YOUR RESULTS HERE (SUCCESS/FAIL/REASON(If fail))\\ \hline
	\end{tabular}
\end{table}

\subsection{Integration Tests}
		%INTEGRATION TEST
	\begin{table}[H]
		\centering
		\caption{Set Speed and Authority}
		\begin{tabular}{|l|l|}
			\hline
			Task & \multicolumn{1}{c|}{Test Design} \\ \hline
			Integrity Level & 3 \\ \hline
			Methods & Evaluate setSpeedAuth() function\\ \hline
			Inputs &  \parbox[t]{10cm}{Block to assign Speed \& Authority to, Authority (as a Block), Speed} \\ \hline
			Outputs & None\\ \hline
			Expected Completion & Set indicated Speed and Authority of specified Block \\ \hline
			Risks and Assumptions & \parbox[t]{10cm}{Block is open and given Speed \& Authority are valid.} \\ \hline
			Responsibility & Wayside Controller\\ \hline
			\\ \hline
			Tested By   &  Max Reno\\	\hline
			Date Tested & \parbox[t]{10cm}{April 19th}\\ \hline
			Results & FILL IN YOUR RESULTS HERE (SUCCESS/FAIL/REASON(If fail))\\ \hline
		\end{tabular}
	\end{table}
	
	%INTEGRATION TEST
	\begin{table}[H]
		\centering
		\caption{Close a Block}
		\begin{tabular}{|l|l|}
			\hline
			Task & \multicolumn{1}{c|}{Test Design} \\ \hline
			Integrity Level & 2 \\ \hline
			Methods & Evaluate closeBlock() function\\ \hline
			Inputs &  Block to be closed\\ \hline
			Outputs & None\\ \hline
			Expected Completion & Track block set to broken status before next state \\ \hline
			Risks and Assumptions & \parbox[t]{10cm}{Block is not already closed and/or occupied} \\ \hline
			Responsibility & Wayside Controller\\ \hline
			\\ \hline
			Tested By   &  Max Reno\\	\hline
			Date Tested & \parbox[t]{10cm}{April 19th}\\ \hline
			Results & FILL IN YOUR RESULTS HERE (SUCCESS/FAIL/REASON(If fail))\\ \hline
		\end{tabular}
	\end{table}
	
	
	%INTEGRATION TEST
	\begin{table}[H]
		\centering
		\caption{Change Switch}
		\begin{tabular}{|l|l|}
			\hline
			Task & \multicolumn{1}{c|}{Test Design} \\ \hline
			Integrity Level & 3 \\ \hline
			Methods & Test PLC code and setSwitchState()\\ \hline
			Inputs &  Block containing desired switch\\ \hline
			Outputs &  True/False result of action completed\\ \hline
			Expected Completion & Switch state is opposite of original state. \\ \hline
			Risks and Assumptions & \parbox[t]{10cm}{Block passed to function contains a valid switch, and that block is not occupied.} \\ \hline
			Responsibility & Wayside Controller\\ \hline
			\\ \hline
			Tested By   &  Max Reno\\	\hline
			Date Tested & \parbox[t]{10cm}{April 19th}\\ \hline
			Results & FILL IN YOUR RESULTS HERE (SUCCESS/FAIL/REASON(If fail))\\ \hline
		\end{tabular}
	\end{table}
	
	%INTEGRATION TEST
	\begin{table}[H]
		\centering
		\caption{Manually Change Switch}
		\begin{tabular}{|l|l|}
			\hline
			Task & \multicolumn{1}{c|}{Test Design} \\ \hline
			Integrity Level & 1 \\ \hline
			Methods & Evaluate manualSwitch() function and setSwitchState()\\ \hline
			Inputs &  Block containing desired switch\\ \hline
			Outputs &  True/False result of action completed\\ \hline
			Expected Completion & Switch state is opposite of original state. \\ \hline
			Risks and Assumptions & \parbox[t]{10cm}{Block passed to function contains a valid switch, and that block is not occupied.} \\ \hline
			Responsibility & Wayside Controller\\ \hline
			\\ \hline
			Tested By   &  Max Reno\\	\hline
			Date Tested & \parbox[t]{10cm}{April 19th}\\ \hline
			Results & FILL IN YOUR RESULTS HERE (SUCCESS/FAIL/REASON(If fail))\\ \hline
		\end{tabular}
	\end{table}

\section{CTC Test Plan}
\subsection{Unit Tests}
%UNIT TEST
\begin{table}[H]
	\centering
	\caption{Choosing Modes}
	\begin{tabular}{|l|l|}
		\hline
		Task & \multicolumn{1}{c|}{Test Design} \\ \hline
		Integrity Level & 1 \\ \hline
		Methods & \parbox[t]{10cm}{Evaluate single radio button selection. Disable Automatic options when Manual chosen, diable dispatch train ability when Automatic chosen.}\\ \hline
		Inputs &  Click on radio buttons. \\ \hline
		Outputs &  See option choice on screen.\\ \hline
		Expected Completion & \parbox[t]{10cm}{With user selection, however Manual is initially chosen at startup.}\\ \hline
		Risks and Assumptions & \parbox[t]{10cm}{Assume only one or the other can be chosen. i.e. can only choose Auto or Manual, not both.} \\ \hline
		Responsibility & CTC\\ \hline
		\\ \hline
		Tested By   &  Christen Reinbeck\\	\hline
		Date Tested & \parbox[t]{10cm}{April 19th}\\ \hline
		Results & FILL IN YOUR RESULTS HERE (SUCCESS/FAIL/REASON(If fail))\\ \hline
	\end{tabular}
\end{table}

%UNIT TEST
\begin{table}[H]
	\centering
	\caption{View Block info}
	\begin{tabular}{|l|l|}
		\hline
		Task & \multicolumn{1}{c|}{Test Design} \\ \hline
		Integrity Level & 2 \\ \hline
		Methods & Tests dropdown functions as well as display of track info.\\ \hline
		Inputs &  Selections of Line, Section, Block \\ \hline
		Outputs &  Info from Excel as well as updates from Wayside.\\ \hline
		Expected Completion & \parbox[t]{10cm}{Upon user selection. Should selection stay on screen, will continue to be updated with time.}\\ \hline
		Risks and Assumptions & \parbox[t]{10cm}{Pulling info from valid CSV file.} \\ \hline
		Responsibility & CTC\\ \hline
		\\ \hline
		Tested By   &  Christen Reinbeck\\	\hline
		Date Tested & \parbox[t]{10cm}{April 19th}\\ \hline
		Results & FILL IN YOUR RESULTS HERE (SUCCESS/FAIL/REASON(If fail))\\ \hline
	\end{tabular}
\end{table}

%UNIT TEST
\begin{table}[H]
	\centering
	\caption{Failure Color Change}
	\begin{tabular}{|l|l|}
		\hline
		Task & \multicolumn{1}{c|}{Test Design} \\ \hline
		Integrity Level & 1 \\ \hline
		Methods & Evaluate color change in Failure Area.\\ \hline
		Inputs &  Failure alerted to CTC. \\ \hline
		Outputs &  Color change occurs on GUI. \\ \hline
		Expected Completion & \parbox[t]{10cm}{Upon receipt of a failure, will pass fake failure to test.}\\ \hline
		Risks and Assumptions & \parbox[t]{10cm}{Assume will only show red or green.} \\ \hline
		Responsibility & CTC\\ \hline
		\\ \hline
		Tested By   &  Christen Reinbeck\\	\hline
		Date Tested & \parbox[t]{10cm}{April 19th}\\ \hline
		Results & FILL IN YOUR RESULTS HERE (SUCCESS/FAIL/REASON(If fail))\\ \hline
	\end{tabular}
\end{table}
\subsection{Integration Tests}

 
%INTEGRATION TEST
\begin{table}[H]
	\centering
	\caption{Dispatch/Edit Train via Button}
	\begin{tabular}{|l|l|}
		\hline
		Task & \multicolumn{1}{c|}{Test Design} \\ \hline
		Integrity Level & 3 \\ \hline
		Methods & \parbox[t]{10cm}{Evaluate the ability to dispatch a train and add it to the list of trains. This info will be passed on to all modules in some way to make/edit a train.}\\ \hline
		Inputs &  \parbox[t]{10cm}{Select Dispatch/Edit Train Button. Complete all info in the resulting popup window (speed, auth, line, id). Click Complete.} \\ \hline
		Outputs &  \parbox[t]{10cm}{Will update the train list displayed to dispatcher as well as the selections to edit.} \\ \hline
		Expected Completion & At any time, at the will of the dispatcher. \\ \hline
		Risks and Assumptions & \parbox[t]{10cm}{Correct occupancy/position data received from Wayside.}  \\ \hline
		Responsibility & CTC\\ \hline
		\\ \hline
		Tested By   &  Christen Reinbeck\\	\hline
		Date Tested & \parbox[t]{10cm}{April 19th}\\ \hline
		Results & FILL IN YOUR RESULTS HERE (SUCCESS/FAIL/REASON(If fail))\\ \hline
	\end{tabular}
\end{table}

%INTEGRATION TEST
\begin{table}[H]
	\centering
	\caption{Close Block/Send Maintenance}
	\begin{tabular}{|l|l|}
		\hline
		Task & \multicolumn{1}{c|}{Test Design} \\ \hline
		Integrity Level & 2 \\ \hline
		Methods & \parbox[t]{10cm}{Evaluate ability to send command to close/repair a block to the wayside.}\\ \hline
		Inputs &  Select correct block, select Close Block or Send Maintenance. \\ \hline
		Outputs &  \parbox[t]{10cm}{Show rerouting/stopping/restarting of trains in train list based on choice.}\\ \hline
		Expected Completion & After a failure is reported.\\ \hline
		Risks and Assumptions & \parbox[t]{10cm}{Failure is reported correctly.} \\ \hline
		Responsibility & CTC\\ \hline
		\\ \hline
		Tested By   &  Christen Reinbeck\\	\hline
		Date Tested & \parbox[t]{10cm}{April 19th}\\ \hline
		Results & FILL IN YOUR RESULTS HERE (SUCCESS/FAIL/REASON(If fail))\\ \hline
	\end{tabular}
\end{table}

%INTEGRATION TEST
\begin{table}[H]
	\centering
	\caption{View Schedule}
	\begin{tabular}{|l|l|}
		\hline
		Task & \multicolumn{1}{c|}{Test Design} \\ \hline
		Integrity Level & 1 \\ \hline
		Methods & Display schedule from MBO in table format.\\ \hline
		Inputs &  Updated schedule from MBO. \\ \hline
		Outputs &  Table of schedules for both lines in popup window. \\ \hline
		Expected Completion & \parbox[t]{10cm}{Whenever dispatcher chooses to view it, and updates occur in time as they happen.}\\ \hline
		Risks and Assumptions & \parbox[t]{10cm}{Valid schedule is passed/correctly updated by MBO.} \\ \hline
		Responsibility & CTC\\ \hline
		\\ \hline
		Tested By   &  Christen Reinbeck\\	\hline
		Date Tested & \parbox[t]{10cm}{April 19th}\\ \hline
		Results & FILL IN YOUR RESULTS HERE (SUCCESS/FAIL/REASON(If fail))\\ \hline
	\end{tabular}
\end{table}

\section{Train Model Test Plan}
\subsection{Unit Tests}
 
    \begin{table}[H]
        \centering
        \caption{Base Test A: Compute Velocity of train at rest with Power command of 100kW}
        \begin{tabular}{|l|l|}
            \hline
            Task & \multicolumn{1}{c|}{Test Design} \\ \hline
            Integrity Level & 2 \\ \hline
            Methods & Apply power command to train and compute velocity  \\ \hline
            Inputs &  Power Command input and "Start Test" button is pressed \\ \hline
            Outputs &  Velocity greater than 0 MPH will be produced  \\ \hline
            Expected Completion & \parbox[t]{10cm}{Test to be performed upon completion of complete submodule.\\ Expected date: March 24th}\\ \hline
            Risks and Assumptions & \parbox[t]{10cm}{The power command should be 100kW for this base case. \\ Assumption will be made that for base test train starts with 0 velocity} \\ \hline
            Responsibility & Train Model\\ \hline
            \\ \hline
            Tested By   &  Demetri Khoury\\	\hline
            Date Tested & \parbox[t]{10cm}{April 12th}\\ \hline
            Results & Success\\ \hline
        \end{tabular}
    \end{table}

	\begin{table}[H]
		\centering
		\caption{Repeat Base Test A with Power command greater than 100,000W}
		\begin{tabular}{|l|l|}
			\hline
			Task & \multicolumn{1}{c|}{Test Design} \\ \hline
			Integrity Level & 2 \\ \hline
			Methods & Apply higher power command to train and compute velocity  \\ \hline
			Inputs &  Power Command input and "Start Test" button is pressed \\ \hline
			Outputs &  Velocity greater than base case A will be produced  \\ \hline
			Expected Completion & \parbox[t]{10cm}{Test to be performed upon completion of complete submodule.\\ Expected date: March 24th}\\ \hline
			Risks and Assumptions & The power command should be a positive value greater than 100kW \\ \hline
			Responsibility & Train Model\\ \hline
			\\ \hline
			Tested By   &  Demetri Khoury\\	\hline
			Date Tested & \parbox[t]{10cm}{April 12th}\\ \hline
			Results & Success\\ \hline
		\end{tabular}
	\end{table}
    
    \begin{table}[H]
    	\centering
    	\caption{Repeat base test A with Power command less than 100,000W}
    	\begin{tabular}{|l|l|}
    		\hline
    		Task & \multicolumn{1}{c|}{Test Design} \\ \hline
    		Integrity Level & 2 \\ \hline
    		Methods & Apply lower power command to train and compute velocity  \\ \hline
    		Inputs &  Power Command input and "Start Test" button is pressed \\ \hline
    		Outputs &  Velocity lower than base case A will be produced  \\ \hline
    		Expected Completion & \parbox[t]{10cm}{Test to be performed upon completion of complete submodule.\\ Expected date: March 24th}\\ \hline
    		Risks and Assumptions & The power command should be a positive value smaller than 100kW \\ \hline
    		Responsibility & Train Model\\ \hline
    		\\ \hline
    		Tested By   &  Demetri Khoury\\	\hline
    		Date Tested & \parbox[t]{10cm}{April 12th}\\ \hline
    		Results &Success\\ \hline
    	\end{tabular}
    \end{table}

	\begin{table}[H]
		\centering
		\caption{Repeat base test A with grade of 3\%}
		\begin{tabular}{|l|l|}
			\hline
			Task & \multicolumn{1}{c|}{Test Design} \\ \hline
			Integrity Level & 2 \\ \hline
			Methods & Increase grade to 3\% and compute velocity  \\ \hline
			Inputs &  Grade set to 3\% and "Start Test" button is pressed \\ \hline
			Outputs &  Velocity lower than base case A will be produced  \\ \hline
			Expected Completion & \parbox[t]{10cm}{Test to be performed upon completion of complete submodule.\\ Expected date: March 24th}\\ \hline
			Risks and Assumptions & \parbox[t]{10cm}{The power command should be equal to 100kW \\and grade will be set to 3\% }\\ \hline
			Responsibility & Train Model\\ \hline
			\\ \hline
			Tested By   &  Demetri Khoury\\	\hline
			Date Tested & \parbox[t]{10cm}{April 12th}\\ \hline
			Results & Success\\ \hline
		\end{tabular}
	\end{table}

	\begin{table}[H]
		\centering
		\caption{Repeat base test A with grade of -3\%}
		\begin{tabular}{|l|l|}
			\hline
			Task & \multicolumn{1}{c|}{Test Design} \\ \hline
			Integrity Level & 2 \\ \hline
			Methods & Decrease grade to -3\% and compute velocity  \\ \hline
			Inputs &  Grade set to -3\% and "Start Test" button is pressed \\ \hline
			Outputs &  Velocity greater than base case A will be produced  \\ \hline
			Expected Completion & \parbox[t]{10cm}{Test to be performed upon completion of complete submodule.\\ Expected date: March 24th}\\ \hline
			Risks and Assumptions &\parbox[t]{10cm}{ The power command should be equal to 100kW \\and grade will be set to -3\%} \\ \hline
			Responsibility & Train Model\\ \hline
			\\ \hline
			Tested By   &  Demetri Khoury\\	\hline
			Date Tested & \parbox[t]{10cm}{April 12th}\\ \hline
			Results & Success\\ \hline
		\end{tabular}
	\end{table}
   
   	\begin{table}[H]
	   	\centering
	   	\caption{Repeat base test A with 150 passengers added}
	   	\begin{tabular}{|l|l|}
	   		\hline
	   		Task & \multicolumn{1}{c|}{Test Design} \\ \hline
	   		Integrity Level & 2 \\ \hline
	   		Methods & Increase passenger count to 150 and compute velocity  \\ \hline
	   		Inputs &  Number of passengers = 150 and "Start Test" button is pressed \\ \hline
	   		Outputs &  Velocity smaller than base case A will be produced  \\ \hline
	   		Expected Completion & \parbox[t]{10cm}{Test to be performed upon completion of complete submodule.\\ Expected date: March 24th}\\ \hline
	   		Risks and Assumptions & \parbox[t]{10cm}{The power command should be equal to 100k0W \\and 150 passengers will be added onboard the train} \\ \hline
	   		Responsibility & Train Model\\ \hline
	   		\\ \hline
	   		Tested By   &  Demetri Khoury\\	\hline
	   		Date Tested & \parbox[t]{10cm}{April 12th}\\ \hline
	   		Results & Success\\ \hline
	   	\end{tabular}
   \end{table}

	\begin{table}[H]
		\centering
		\caption{Base Case B: Compute Velocity of train at 25MPH with Power command of 100kW }
		\begin{tabular}{|l|l|}
			\hline
			Task & \multicolumn{1}{c|}{Test Design} \\ \hline
			Integrity Level & 2 \\ \hline
			Methods & Apply Power command of 100kW and compute new velocity  \\ \hline
			Inputs &  Power Command set to 100kW and "Start Test" button is pressed \\ \hline
			Outputs &  Velocity larger than 25MPH will be produced  \\ \hline
			Expected Completion & \parbox[t]{10cm}{Test to be performed upon completion of complete submodule.\\ Expected date: March 24th}\\ \hline
			Risks and Assumptions & The power command should be equal to 100kW \\ \hline
			Responsibility & Train Model\\ \hline
			\\ \hline
			Tested By   &  Demetri Khoury\\	\hline
			Date Tested & \parbox[t]{10cm}{April 12th}\\ \hline
			Results & Success\\ \hline
		\end{tabular}
	\end{table}

	\begin{table}[H]
		\centering
		\caption{Repeat Base Case B with power command of 0W }
		\begin{tabular}{|l|l|}
			\hline
			Task & \multicolumn{1}{c|}{Test Design} \\ \hline
			Integrity Level & 2 \\ \hline
			Methods & Apply Power command of 0W and compute new velocity  \\ \hline
			Inputs &  Power Command set to 0W and "Start Test" button is pressed \\ \hline
			Outputs &  Velocity smaller than 25MPH will be produced  \\ \hline
			Expected Completion & \parbox[t]{10cm}{Test to be performed upon completion of complete submodule.\\ Expected date: March 24th}\\ \hline
			Risks and Assumptions & The power command should be equal to 0W \\ \hline
			Responsibility & Train Model\\ \hline
			\\ \hline
			Tested By   &  Demetri Khoury\\	\hline
			Date Tested & \parbox[t]{10cm}{April 12th}\\ \hline
			Results & Success\\ \hline
		\end{tabular}
	\end{table}

	\begin{table}[H]
		\centering
		\caption{Repeat Base Case B with power command less than 0W }
		\begin{tabular}{|l|l|}
			\hline
			Task & \multicolumn{1}{c|}{Test Design} \\ \hline
			Integrity Level & 2 \\ \hline
			Methods & Apply negative power command and compute new velocity  \\ \hline
			Inputs &  Power Command set to -100kW and "Start Test" button is pressed \\ \hline
			Outputs &  Invalid input message will appear  \\ \hline
			Expected Completion & \parbox[t]{10cm}{Test to be performed upon completion of complete submodule.\\ Expected date: March 24th}\\ \hline
			Risks and Assumptions & Power command must be positive for all possible cases\\ \hline
			\\ \hline
			Tested By   &  Demetri Khoury\\	\hline
			Date Tested & \parbox[t]{10cm}{April 12th}\\ \hline
			Results & Success\\ \hline
		\end{tabular}
	\end{table}

	\begin{table}[H]
		\centering
		\caption{Repeat Base Case B with power command greater than 120kW }
		\begin{tabular}{|l|l|}
			\hline
			Task & \multicolumn{1}{c|}{Test Design} \\ \hline
			Integrity Level & 2 \\ \hline
			Methods & Apply power command above max and compute new velocity  \\ \hline
			Inputs &  Power Command set to 150kW and "Start Test" button is pressed \\ \hline
			Outputs &  Speed will remain 25MPH as there is no more power available  \\ \hline
			Expected Completion & \parbox[t]{10cm}{Test to be performed upon completion of complete submodule.\\ Expected date: March 24th}\\ \hline
			Risks and Assumptions & If power exceeds max, the velocity stays the same\\ \hline
			\\ \hline
			Tested By   &  Demetri Khoury\\	\hline
			Date Tested & \parbox[t]{10cm}{April 12th}\\ \hline
			Results & Success\\ \hline
		\end{tabular}
	\end{table}

	\begin{table}[H]
		\centering
		\caption{Repeat Base Case B but apply Service brakes}
		\begin{tabular}{|l|l|}
			\hline
			Task & \multicolumn{1}{c|}{Test Design} \\ \hline
			Integrity Level & 3 \\ \hline
			Methods & Apply service brakes and compute new velocity  \\ \hline
			Inputs &  Service brakes engaged and "Start Test" button is pressed \\ \hline
			Outputs & \parbox[t]{10cm}{ Service brake status switched to ON\\ Power Command set to 0\\ Train speed decreased to lower than 25 MPH  }\\ \hline
			Expected Completion & \parbox[t]{10cm}{Test to be performed upon completion of complete submodule.\\ Expected date: March 24th}\\ \hline
			Risks and Assumptions & Service brake will automatically override power command to 0W\\ \hline
			\\ \hline
			Tested By   &  Demetri Khoury\\	\hline
			Date Tested & \parbox[t]{10cm}{April 12th}\\ \hline
			Results & Success\\ \hline
		\end{tabular}
	\end{table}

	\begin{table}[H]
		\centering
		\caption{Repeat Base Case B but apply Emergency brakes}
		\begin{tabular}{|l|l|}
			\hline
			Task & \multicolumn{1}{c|}{Test Design} \\ \hline
			Integrity Level & 3 \\ \hline
			Methods & Apply Emergency brakes and compute new velocity  \\ \hline
			Inputs &  Emergency brakes engaged and "Start Test" button is pressed \\ \hline
			Outputs & \parbox[t]{10cm}{ Emergency brake status switched to ON\\ Power Command set to 0\\ Train speed decreased to lower than 25 MPH\\ Train speed also lower than service brake test case  }\\ \hline
			Expected Completion & \parbox[t]{10cm}{Test to be performed upon completion of complete submodule.\\ Expected date: March 24th}\\ \hline
			Risks and Assumptions & Service brake will automatically override power command to 0W\\ \hline
			\\ \hline
			Tested By   &  Demetri Khoury\\	\hline
			Date Tested & \parbox[t]{10cm}{April 19th}\\ \hline
			Results & Success\\ \hline
		\end{tabular}
	\end{table}

	\begin{table}[H]
		\centering
		\caption{Activate engine failure on moving train}
		\begin{tabular}{|l|l|}
			\hline
			Task & \multicolumn{1}{c|}{Test Design} \\ \hline
			Integrity Level & 2 \\ \hline
			Methods & Toggle engine failure status to on  \\ \hline
			Inputs &  Radio button for engine failure set to ON \\ \hline
			Outputs & \parbox[t]{10cm}{ Engine Failure status switched to ON\\ Power Command set to 0\\ Train speed decreased to 0 MPH\\ Service brake status set to ON  }\\ \hline
			Expected Completion & \parbox[t]{10cm}{Test to be performed upon completion of complete submodule.\\ Expected date: April 5th}\\ \hline
			Risks and Assumptions & Service brake will automatically activate on failure\\ \hline
			\\ \hline
			Tested By   &  Demetri Khoury\\	\hline
			Date Tested & \parbox[t]{10cm}{April 19th}\\ \hline
			Results & Success\\ \hline
		\end{tabular}
	\end{table}

	\begin{table}[H]
		\centering
		\caption{Activate Signal failure on moving train}
		\begin{tabular}{|l|l|}
			\hline
			Task & \multicolumn{1}{c|}{Test Design} \\ \hline
			Integrity Level & 2 \\ \hline
			Methods & Toggle signal failure status to on  \\ \hline
			Inputs &  Radio button for signal failure set to ON \\ \hline
			Outputs & \parbox[t]{10cm}{ Signal Failure status switched to ON\\ Power Command set to 0\\ Train speed decreased to 0 MPH\\ Service brake status set to ON  }\\ \hline
			Expected Completion & \parbox[t]{10cm}{Test to be performed upon completion of complete submodule.\\ Expected date: April 5th}\\ \hline
			Risks and Assumptions & Service brake will automatically activate on failure\\ \hline
			\\ \hline
			Tested By   &  Demetri Khoury\\	\hline
			Date Tested & \parbox[t]{10cm}{April 19th}\\ \hline
			Results & Success\\ \hline
		\end{tabular}
	\end{table}

	\begin{table}[H]
		\centering
		\caption{Activate Brake failure on moving train}
		\begin{tabular}{|l|l|}
			\hline
			Task & \multicolumn{1}{c|}{Test Design} \\ \hline
			Integrity Level & 2 \\ \hline
			Methods & Toggle brake failure status to on  \\ \hline
			Inputs &  Radio button for brake failure set to ON \\ \hline
			Outputs & \parbox[t]{10cm}{ Brake Failure status switched to ON\\ Power Command set to 0\\ Train speed decreased to 0 MPH\\ Emergency brake status set to ON  }\\ \hline
			Expected Completion & \parbox[t]{10cm}{Test to be performed upon completion of complete submodule.\\ Expected date: April 5th}\\ \hline
			Risks and Assumptions & Emergency brake will activate on failure in service brakes\\ \hline
			\\ \hline
			Tested By   &  Demetri Khoury\\	\hline
			Date Tested & \parbox[t]{10cm}{April 19th}\\ \hline
			Results & Success\\ \hline
		\end{tabular}
	\end{table}

	\begin{table}[H]
		\centering
		\caption{Open doors on moving train}
		\begin{tabular}{|l|l|}
			\hline
			Task & \multicolumn{1}{c|}{Test Design} \\ \hline
			Integrity Level & 2 \\ \hline
			Methods & Open left side doors on moving train  \\ \hline
			Inputs &  Radio button for left doors set to OPEN \\ \hline
			Outputs & Invalid action pop-up\\ \hline
			Expected Completion & \parbox[t]{10cm}{Test to be performed upon completion of complete submodule.\\ Expected date: April 5th}\\ \hline
			Risks and Assumptions & Doors will not open while train is in motion\\ \hline
			\\ \hline
			Tested By   &  Demetri Khoury\\	\hline
			Date Tested & \parbox[t]{10cm}{April 19th}\\ \hline
			Results & Success\\ \hline
		\end{tabular}
	\end{table}

	\begin{table}[H]
		\centering
		\caption{Open doors on non-moving train}
		\begin{tabular}{|l|l|}
			\hline
			Task & \multicolumn{1}{c|}{Test Design} \\ \hline
			Integrity Level & 2 \\ \hline
			Methods & Open left side doors on non-moving train  \\ \hline
			Inputs &  Radio button for left doors set to OPEN \\ \hline
			Outputs & Left door status on Train model changes to OPEN\\ \hline
			Expected Completion & \parbox[t]{10cm}{Test to be performed upon completion of complete submodule.\\ Expected date: April 5th}\\ \hline
			Risks and Assumptions &\parbox[t]{10cm}{ Left and Right doors can both be opened at the same time \\ but opening is independent}\\ \hline
			\\ \hline
			Tested By   &  Demetri Khoury\\	\hline
			Date Tested & \parbox[t]{10cm}{April 19th}\\ \hline
			Results & Success\\ \hline
		\end{tabular}
	\end{table}

	\begin{table}[H]
		\centering
		\caption{Power Command applied to non-moving train with open doors}
		\begin{tabular}{|l|l|}
			\hline
			Task & \multicolumn{1}{c|}{Test Design} \\ \hline
			Integrity Level & 2 \\ \hline
			Methods & Open left door, then apply power of 100kW  \\ \hline
			Inputs &  \parbox[t]{10cm}{Radio button for left doors set to OPEN\\ Power command Set to 100kW\\ "Start Test" button pressed }\\ \hline
			Outputs & Error message pop-up\\ \hline
			Expected Completion & \parbox[t]{10cm}{Test to be performed upon completion of complete submodule.\\ Expected date: April 5th}\\ \hline
			Risks and Assumptions & Train can not move if doors are open\\ \hline
			\\ \hline
			Tested By   &  Demetri Khoury\\	\hline
			Date Tested & \parbox[t]{10cm}{April 19th}\\ \hline
			Results & Success : train controller forces doors to close before moving\\ \hline
		\end{tabular}
	\end{table}

	\begin{table}[H]
		\centering
		\caption{Power Command applied to non-moving train with Failure status}
		\begin{tabular}{|l|l|}
			\hline
			Task & \multicolumn{1}{c|}{Test Design} \\ \hline
			Integrity Level & 2 \\ \hline
			Methods & Engage any failure, then apply power of 100kW  \\ \hline
			Inputs &  \parbox[t]{10cm}{Radio button for Engine Failure set to ON\\ Power command Set to 100kW\\ "Start Test" button pressed }\\ \hline
			Outputs & Error message pop-up\\ \hline
			Expected Completion & \parbox[t]{10cm}{Test to be performed upon completion of complete submodule.\\ Expected date: April 5th}\\ \hline
			Risks and Assumptions & Train can not move if failures are present\\ \hline
			\\ \hline
			Tested By   &  Demetri Khoury\\	\hline
			Date Tested & \parbox[t]{10cm}{April 19th}\\ \hline
			Results & Success\\ \hline
		\end{tabular}
	\end{table}

	\begin{table}[H]
		\centering
		\caption{Power Command applied to non-moving train with engaged brakes}
		\begin{tabular}{|l|l|}
			\hline
			Task & \multicolumn{1}{c|}{Test Design} \\ \hline
			Integrity Level & 2 \\ \hline
			Methods & Engage either brake, then apply power of 100kW  \\ \hline
			Inputs &  \parbox[t]{10cm}{Service Brakes engaged\\ Power command Set to 100kW\\ "Start Test" button pressed }\\ \hline
			Outputs & Error message pop-up\\ \hline
			Expected Completion & \parbox[t]{10cm}{Test to be performed upon completion of complete submodule.\\ Expected date: April 5th}\\ \hline
			Risks and Assumptions & Train can not move if brakes are engaged\\ \hline
			\\ \hline
			Tested By   &  Demetri Khoury\\	\hline
			Date Tested & \parbox[t]{10cm}{April 19th}\\ \hline
			Results & Success\\ \hline
		\end{tabular}
	\end{table}

	\begin{table}[H]
		\centering
		\caption{Interior Light Test}
		\begin{tabular}{|l|l|}
			\hline
			Task & \multicolumn{1}{c|}{Test Design} \\ \hline
			Integrity Level & 1 \\ \hline
			Methods & Lights turned on in test console  \\ \hline
			Inputs &  \parbox[t]{10cm}{Radio button for lights set to ON\\ "Start Test" button pressed }\\ \hline
			Outputs &Interior Lights set to ON \\ \hline
			Expected Completion & \parbox[t]{10cm}{Test to be performed upon completion of complete submodule.\\ Expected date: April 5th}\\ \hline
			Risks and Assumptions & Lights can be on at any time.\\ \hline
			\\ \hline
			Tested By   &  Demetri Khoury\\	\hline
			Date Tested & \parbox[t]{10cm}{April 19th}\\ \hline
			Results & Success\\ \hline
		\end{tabular}
	\end{table}

	\begin{table}[H]
		\centering
		\caption{Train Temperature set to 60F Thermostat set to 65F}
		\begin{tabular}{|l|l|}
			\hline
			Task & \multicolumn{1}{c|}{Test Design} \\ \hline
			Integrity Level & 1 \\ \hline
			Methods & Set intial temp to 60F, Set Thermostat to 65F  \\ \hline
			Inputs &  \parbox[t]{10cm}{Train Temperature set to 60F\\ Thermostat set to 65F\\ "Start Test" button pressed }\\ \hline
			Outputs &\parbox[t]{10cm}{ Heater set to ON\\AC set to OFF\\ Temperature increases to 65F} \\ \hline
			Expected Completion & \parbox[t]{10cm}{Test to be performed upon completion of complete submodule.\\ Expected date: April 5th}\\ \hline
			Risks and Assumptions & Heater and AC can not be on at the same time.\\ \hline
			\\ \hline
			Tested By   &  Demetri Khoury\\	\hline
			Date Tested & \parbox[t]{10cm}{April 19th}\\ \hline
			Results & Success\\ \hline
		\end{tabular}
	\end{table}



	\begin{table}[H]
		\centering
		\caption{Train Temperature set to 60F Thermostat set to 60F}
		\begin{tabular}{|l|l|}
			\hline
			Task & \multicolumn{1}{c|}{Test Design} \\ \hline
			Integrity Level & 1 \\ \hline
			Methods & Set intial temp to 60F, Set Thermostat to 60F  \\ \hline
			Inputs &  \parbox[t]{10cm}{Train Temperature set to 60F\\ Thermostat set to 60F\\ "Start Test" button pressed }\\ \hline
			Outputs &\parbox[t]{10cm}{ Heater set to OFF\\AC set to OFF\\ Temperature does not change} \\ \hline
			Expected Completion & \parbox[t]{10cm}{Test to be performed upon completion of complete submodule.\\ Expected date: April 5th}\\ \hline
			Risks and Assumptions & \parbox[t]{10cm}{Temperature can only change if heat or AC is on\\ No heat loss due to windows open}\\ \hline
			\\ \hline
			Tested By   &  Demetri Khoury\\	\hline
			Date Tested & \parbox[t]{10cm}{April 19th}\\ \hline
			Results & Success\\ \hline
		\end{tabular}
	\end{table}

	\begin{table}[H]
		\centering
		\caption{Train Temperature set to 60F Thermostat set to 55F}
		\begin{tabular}{|l|l|}
			\hline
			Task & \multicolumn{1}{c|}{Test Design} \\ \hline
			Integrity Level & 1 \\ \hline
			Methods & Set intial temp to 60F, Set Thermostat to 55F  \\ \hline
			Inputs &  \parbox[t]{10cm}{Train Temperature set to 60F\\ Thermostat set to 55F\\ "Start Test" button pressed }\\ \hline
			Outputs &\parbox[t]{10cm}{ Heater set to OFF\\AC set to ON\\ Temperature decreases to 55F} \\ \hline
			Expected Completion & \parbox[t]{10cm}{Test to be performed upon completion of complete submodule.\\ Expected date: April 5th}\\ \hline
			Risks and Assumptions & Heater and AC can not be on at the same time.\\ \hline
			\\ \hline
			Tested By   &  Demetri Khoury\\	\hline
			Date Tested & \parbox[t]{10cm}{April 19th}\\ \hline
			Results & Success\\ \hline
		\end{tabular}
	\end{table}

	\begin{table}[H]
		\centering
		\caption{Integration test With train Controller}
		\begin{tabular}{|l|l|}
			\hline
			Task & \multicolumn{1}{c|}{Test Design} \\ \hline
			Integrity Level & 1 \\ \hline
			Methods & Repeat above tests but receiving values from Train Controller \\ \hline
			Inputs &  Power Command, Utility statuses, Brake Statuses\\ \hline
			Outputs & Outputs should reflect similar results as the test cases above \\ \hline
			Expected Completion & \parbox[t]{10cm}{Test to be performed upon completion of integration with train controller.\\ Expected date: April 7th}\\ \hline
			Risks and Assumptions & \parbox[t]{10cm}{Whether inputs come from train controller or test console results should be the same}\\ \hline
				\\ \hline
			Tested By   &  Demetri Khoury\\	\hline
			Date Tested & \parbox[t]{10cm}{April 19th}\\ \hline
			Results &Success\\ \hline
		\end{tabular}
	\end{table}

	\begin{table}[H]
		\centering
		\caption{Integration test With Track Model}
		\begin{tabular}{|l|l|}
			\hline
			Task & \multicolumn{1}{c|}{Test Design} \\ \hline
			Integrity Level & 1 \\ \hline
			Methods & Read in values for current block's grade for calculations \\ \hline
			Inputs & Request for current grade to track Model\\ \hline
			Outputs & If successful a grade will be returned to train model \\ \hline
			Expected Completion & \parbox[t]{10cm}{Test to be performed upon completion of integration with Track Model.\\ Expected date: April 7th}\\ \hline
			Risks and Assumptions & Track model will send grade upon entrance to block\\ \hline
				\\ \hline
			Tested By   &  Demetri Khoury\\	\hline
			Date Tested & \parbox[t]{10cm}{April 19th}\\ \hline
			Results & Success\\ \hline
		\end{tabular}
	\end{table}

	\begin{table}[H]
		\centering
		\caption{Integration test With MBO}
		\begin{tabular}{|l|l|}
			\hline
			Task & \multicolumn{1}{c|}{Test Design} \\ \hline
			Integrity Level & 1 \\ \hline
			Methods & Testing communication between MBo and Train model \\ \hline
			Inputs & Request for current location from MBO\\ \hline
			Outputs & If successful a location will be sent to the MBO \\ \hline
			Expected Completion & \parbox[t]{10cm}{Test to be performed upon completion of integration with MBO.\\ Expected date: April 7th}\\ \hline
			Risks and Assumptions & Train model will periodically update MBO with location\\ \hline
				\\ \hline
			Tested By   &  Demetri Khoury\\	\hline
			Date Tested & \parbox[t]{10cm}{April 19th}\\ \hline
			Results & Success\\ \hline
		\end{tabular}
	\end{table}
\subsection{Integration Tests}

\section{Train Controller Test Plan}
Author: Andrew Lendacky
\subsection{Unit Tests}

\begin{table}[H]
	\centering
	\caption{UI Elements Disabled in Automatic Mode}
	\begin{tabular}{|l|l|}
		\hline
		Task & \multicolumn{1}{c|}{Test Design} \\ \hline
		Integrity Level & 2 \\ \hline
		Methods & Checks to see if the desired UI elements are disabled\\ \hline
		Inputs & The variable 'inAutomaticMode'\\ \hline
		Outputs & All the desired elements are disabled. \\ \hline
		Expected Completion & When the system is in Automatic mode\\ \hline
		Risks and Assumptions & The desired elements are known\\ \hline
		Responsibility & Train Controller\\ \hline
			\\ \hline
		Tested By   &  Andrew Lendacky\\	\hline
		Date Tested & \parbox[t]{10cm}{March 20 - April 19th}\\ \hline
		Results & Success. Elements are correctly disabled.\\ \hline
	\end{tabular}
\end{table}

\begin{table}[H]
	\centering
	\caption{System is in Manual Mode}
	\begin{tabular}{|l|l|}
		\hline
		Task & \multicolumn{1}{c|}{Test Design} \\ \hline
		Integrity Level & 3 \\ \hline
		Methods & Compares 'inManualMode' and 'inAutomaticMode'\\ \hline
		Inputs & 'inManualMode' and 'inAutomaticMode', which are booleans\\ \hline
		Outputs & 'inManualMode' is true and 'inAutomaticMode' is false \\ \hline
		Expected Completion & When the system is switched to Manual mode\\ \hline
		Risks and Assumptions & none\\ \hline
		Responsibility & Train Controller\\ \hline
			\\ \hline
		Tested By   &  Andrew Lendacky\\	\hline
		Date Tested & \parbox[t]{10cm}{March 20 - April 19th}\\ \hline
		Results & Success. Mode is successfully set.\\ \hline
	\end{tabular}
\end{table}

\begin{table}[H]
	\centering
	\caption{System is in Automatic Mode}
	\begin{tabular}{|l|l|}
		\hline
		Task & \multicolumn{1}{c|}{Test Design} \\ \hline
		Integrity Level & 3 \\ \hline
		Methods & Compares 'inManualMode' and 'inAutomaticMode'\\ \hline
		Inputs & 'inManualMode' and 'inAutomaticMode, which are booleans'\\ \hline
		Outputs & 'inManualMode' is false and 'inAutomaticMode' is true \\ \hline
		Expected Completion & When the system is switched to Automatic mode\\ \hline
		Risks and Assumptions & none\\ \hline
		Responsibility & Train Controller\\ \hline
			\\ \hline
		Tested By   &  Andrew Lendacky\\	\hline
		Date Tested & \parbox[t]{10cm}{March 20 - April 19th}\\ \hline
		Results & Mode is successfully set.\\ \hline
	\end{tabular}
\end{table}

\begin{table}[H]
	\centering
	\caption{System is in Normal Mode}
	\begin{tabular}{|l|l|}
		\hline
		Task & \multicolumn{1}{c|}{Test Design} \\ \hline
		Integrity Level & 2\\ \hline
		Methods & Compares 'inNormalMode' and 'inTestingMode'\\ \hline
		Inputs & 'inTestingMode' and 'inNormalMode', which are booleans\\ \hline
		Outputs & 'inTestingMode' is false and 'inNormalMode' is true \\ \hline
		Expected Completion & When the system is switched to Normal mode\\ \hline
		Risks and Assumptions & none\\ \hline
		Responsibility & Train Controller\\ \hline
			\\ \hline
		Tested By   &  Andrew Lendacky\\	\hline
		Date Tested & \parbox[t]{10cm}{March 20 - April 19th}\\ \hline
		Results & Mode is successfully set.\\ \hline
	\end{tabular}
\end{table}

\begin{table}[H]
	\centering
	\caption{System is in Testing Mode}
	\begin{tabular}{|l|l|}
		\hline
		Task & \multicolumn{1}{c|}{Test Design} \\ \hline
		Integrity Level & 2 \\ \hline
		Methods & Compares 'inTestingMode' and 'inNormalMode'\\ \hline
		Inputs & 'inTestingMode' and 'inNormalMode', which are booleans\\ \hline
		Outputs & 'inTestingMode' is true and 'inNormalMode'  is false\\ \hline
		Expected Completion & When the system is switched to Testing mode\\ \hline
		Risks and Assumptions & none\\ \hline
		Responsibility & Train Controller\\ \hline
			\\ \hline
		Tested By   &  Andrew Lendacky\\	\hline
		Date Tested & \parbox[t]{10cm}{March 20 - April 19th}\\ \hline
		Results & Mode is successfully set. \\ \hline
	\end{tabular}
\end{table}

\begin{table}[H]
	\centering
	\caption{Set Speed is Not Greater than Block Speed}
	\begin{tabular}{|l|l|}
		\hline
		Task & \multicolumn{1}{c|}{Test Design} \\ \hline
		Integrity Level & 3 \\ \hline
		Methods & Compares the set speed and the block speed\\ \hline
		Inputs & The block speed and the set speed\\ \hline
		Outputs & The set speed is equal to the block speed \\ \hline
		Expected Completion & When the 'Set Speed' button is clicked\\ \hline
		Risks and Assumptions & The system is in Manual mode\\ \hline
		Responsibility & Train Controller\\ \hline
			\\ \hline
		Tested By   &  Andrew Lendacky\\	\hline
		Date Tested & \parbox[t]{10cm}{March 20 - April 19th}\\ \hline
		Results & Success.\\ \hline
	\end{tabular}
\end{table}

\begin{table}[H]
	\centering
	\caption{Set Speed is Not Greater than Suggested Speed}
	\begin{tabular}{|l|l|}
		\hline
		Task & \multicolumn{1}{c|}{Test Design} \\ \hline
		Integrity Level & 3 \\ \hline
		Methods & Compares the set speed and the suggested speed\\ \hline
		Inputs & The suggested speed and the set speed\\ \hline
		Outputs & The set speed is equal to the suggested speed \\ \hline
		Expected Completion & When the train needs to change speeds\\ \hline
		Risks and Assumptions & The system is in Automatic mode \\ \hline
		Responsibility &  Train Controller\\ \hline
			\\ \hline
		Tested By   &  Andrew Lendacky\\	\hline
		Date Tested & \parbox[t]{10cm}{March 20 - April 19th}\\ \hline
		Results & Success\\ \hline
	\end{tabular}
\end{table}

\begin{table}[H]
	\centering
	\caption{Slider\'s Max Value is Equal to the Block Speed Limit}
	\begin{tabular}{|l|l|}
		\hline
		Task & \multicolumn{1}{c|}{Test Design} \\ \hline
		Integrity Level & 3 \\ \hline
		Methods & Compares the slider's max value to the suggested speed\\ \hline
		Inputs & The suggested speed\\ \hline
		Outputs & The suggested speed equals the max value of the slider \\ \hline
		Expected Completion & When the suggested speed is changed\\ \hline
		Risks and Assumptions & The system is in Automatic mode \\ \hline
		Responsibility &  Train Controller\\ \hline
			\\ \hline
		Tested By   &  Andrew Lendacky\\	\hline
		Date Tested & \parbox[t]{10cm}{March 20 - April 19th}\\ \hline
		Results & Success\\ \hline
	\end{tabular}
\end{table}

\subsection{Integration Tests}

\begin{table}[H]
	\centering
	\caption{Selecting a Train}
	\begin{tabular}{|l|l|}
		\hline
		Task & \multicolumn{1}{c|}{Test Design} \\ \hline
		Integrity Level & 3 \\ \hline
		Methods & Compare the IDs of the two trains\\ \hline
		Inputs & ID of the train selected \\ \hline
		Outputs &  The two IDs match\\ \hline
		Expected Completion & When a train is selected from the dropdown menu\\ \hline
		Risks and Assumptions & there is at least one dispatched train\\ \hline
		Responsibility & Train Controller\\ \hline
			\\ \hline
		Tested By   &  Andrew Lendacky\\	\hline
		Date Tested & \parbox[t]{10cm}{March 20 - April 19th}\\ \hline
		Results & Success\\ \hline
	\end{tabular}
\end{table}

\begin{table}[H]
	\centering
	\caption{Turning AC On - Using Radio Button}
	\begin{tabular}{|l|l|}
		\hline
		Task & \multicolumn{1}{c|}{Test Design} \\ \hline
		Integrity Level & 1 \\ \hline
		Methods & Compares the states of the AC and heat on the train\\ \hline
		Inputs & The selected train\\ \hline
		Outputs & The AC is on and the heat is off \\ \hline
		Expected Completion & When the 'ON' radio button is selected\\ \hline
		Risks and Assumptions & System is in Automatic or Manual mode, heat was on \\ \hline
		Responsibility & Train Controller\\ \hline
			\\ \hline
		Tested By   &  Andrew Lendacky\\	\hline
		Date Tested & \parbox[t]{10cm}{March 20 - April 19th}\\ \hline
		Results & Success\\ \hline
	\end{tabular}
\end{table}

\begin{table}[H]
	\centering
	\caption{Turning AC On - Clicking Set}
	\begin{tabular}{|l|l|}
		\hline
		Task & \multicolumn{1}{c|}{Test Design} \\ \hline
		Integrity Level & 1 \\ \hline
		Methods & Compares the states of the AC and heat on the train\\ \hline
		Inputs & The selected train\\ \hline
		Outputs & The AC is on and the heat is off \\ \hline
		Expected Completion & When the 'Set' button is clicked\\ \hline
		Risks and Assumptions & System is in Manual Mode, heat was on \\ \hline
		Responsibility &  Train Controller\\ \hline
			\\ \hline
		Tested By   &  Andrew Lendacky\\	\hline
		Date Tested & \parbox[t]{10cm}{March 20 - April 19th}\\ \hline
		Results & Success\\ \hline
	\end{tabular}
\end{table}

\begin{table}[H]
	\centering
	\caption{Turning Heat On - Using Radio Buttons}
	\begin{tabular}{|l|l|}
		\hline
		Task & \multicolumn{1}{c|}{Test Design} \\ \hline
		Integrity Level & 1 \\ \hline
		Methods & Compares the states of the AC and the heat on the train\\ \hline
		Inputs & The selected train\\ \hline
		Outputs & The heat is on and the AC is off\\ \hline
		Expected Completion & When the 'ON' radio button is selected\\ \hline
		Risks and Assumptions & System is in Automatic or Manual mode, AC was on \\ \hline
		Responsibility &  Train Controller\\ \hline
			\\ \hline
		Tested By   &  Andrew Lendacky\\	\hline
		Date Tested & \parbox[t]{10cm}{March 20 - April 19th}\\ \hline
		Results & Success\\ \hline
	\end{tabular}
\end{table}

\begin{table}[H]
	\centering
	\caption{Turning Heat On - Clicking Set}
	\begin{tabular}{|l|l|}
		\hline
		Task & \multicolumn{1}{c|}{Test Design} \\ \hline
		Integrity Level & 1 \\ \hline
		Methods & Compares the states of the AC and the heat on the train\\ \hline
		Inputs & The selected train\\ \hline
		Outputs & The heat is on and the AC is off \\ \hline
		Expected Completion & When the 'Set' button is clicked\\ \hline
		Risks and Assumptions & System is in Manual mode, AC was on \\ \hline
		Responsibility &  Train Controller\\ \hline
			\\ \hline
		Tested By   &  Andrew Lendacky\\	\hline
		Date Tested & \parbox[t]{10cm}{March 20 - April 19th}\\ \hline
		Results & Success\\ \hline
	\end{tabular}
\end{table}

\begin{table}[H]
	\centering
	\caption{Failures Window Reflects Failure on Train}
	\begin{tabular}{|l|l|}
		\hline
		Task & \multicolumn{1}{c|}{Test Design} \\ \hline
		Integrity Level & 3 \\ \hline
		Methods & Compares the state of the train to the radio buttons\\ \hline
		Inputs & The selected train\\ \hline
		Outputs & The radio buttons match the failure on the train\\ \hline
		Expected Completion & When a failure on the train occurs\\ \hline
		Risks and Assumptions & the test checks all three (antenna, power, and brake) failures \\ \hline
		Responsibility & Train Controller\\ \hline
			\\ \hline
		Tested By   &  Andrew Lendacky\\	\hline
		Date Tested & \parbox[t]{10cm}{March 20 - April 19th}\\ \hline
		Results & Success\\ \hline
	\end{tabular}
\end{table}

\begin{table}[H]
	\centering
	\caption{Sub-Component Receives Correct Train}
	\begin{tabular}{|l|l|}
		\hline
		Task & \multicolumn{1}{c|}{Test Design} \\ \hline
		Integrity Level & 3 \\ \hline
		Methods & Compares the two IDs of the trains\\ \hline
		Inputs & The selected train from the Train Cont.\\ \hline
		Outputs & The two IDs match \\ \hline
		Expected Completion & When a train is selected from the dropdown\\ \hline
		Risks and Assumptions & none \\ \hline
		Responsibility & Train Controller\\ \hline
			\\ \hline
		Tested By   &  Andrew Lendacky\\	\hline
		Date Tested & \parbox[t]{10cm}{March 20 - April 19th}\\ \hline
		Results & Success\\ \hline
	\end{tabular}
\end{table}

\section{MBO Test Plan}
Author: Zach Scheider

\subsection{Unit Tests}

%UNIT TEST
\begin{table}[H]
	\centering
	\caption{Generate Train Schedule}
	\begin{tabular}{|l|l|}
		\hline
		Task & \multicolumn{1}{c|}{Test Design} \\ \hline
		Integrity Level & 3 \\ \hline
		Methods & Generate a schedule so that a train stops at each station. \\ \hline
		Inputs &  Number of trains, info from excel file, block occupancy \\ \hline
		Outputs &  Schedule for trains for each line \\ \hline
		Expected Completion & \parbox[t]{10cm}{April 15th}\\ \hline
		Risks and Assumptions & \parbox[t]{10cm}{Able to receive information from excel at load time.} \\ \hline
		Responsibility & MBO\\ \hline	
		\\ \hline
		Tested By   &  Zach Scheider\\	\hline
		Date Tested & \parbox[t]{10cm}{April 14th}\\ \hline
		Results & Success for red line\\ \hline
	\end{tabular}
\end{table}

%UNIT TEST
\begin{table}[H]
	\centering
	\caption{Update Trains}
	\begin{tabular}{|l|l|}
		\hline
		Task & \multicolumn{1}{c|}{Test Design} \\ \hline
		Integrity Level & 3 \\ \hline
		Methods & \parbox[t]{10cm} {Update trains authority and suggested speed at each clock tick \\in MBO and FB mode.} \\ \hline
		Inputs &  Number of trains, info from excel file, block occupancy \\ \hline
		Outputs &  Updated train speed and authority \\ \hline
		Expected Completion & \parbox[t]{10cm}{April 15th}\\ \hline
		Risks and Assumptions & \parbox[t]{10cm}{Able to receive information from excel at load time.} \\ \hline
		Responsibility & MBO\\ \hline
			\\ \hline
		Tested By   &  Zach Scheider\\	\hline
		Date Tested & \parbox[t]{10cm}{April 15th}\\ \hline
		Results & Success for red line\\ \hline
	\end{tabular}
\end{table}

%UNIT TEST
\begin{table}[H]
	\centering
	\caption{View Schedule}
	\begin{tabular}{|l|l|}
		\hline
		Task & \multicolumn{1}{c|}{Test Design} \\ \hline
		Integrity Level & 1 \\ \hline
		Methods & Display schedule in table format.\\ \hline
		Inputs &  Updated schedule. \\ \hline
		Outputs &  Table of schedules for both lines in popup window. \\ \hline
		Expected Completion & \parbox[t]{10cm}{April 15th}\\ \hline
		Risks and Assumptions & \parbox[t]{10cm}{Valid schedule is passed/correctly updated.} \\ \hline
		Responsibility & MBO\\ \hline
			\\ \hline
		Tested By   &  Zach Scheider\\	\hline
		Date Tested & \parbox[t]{10cm}{April 15th}\\ \hline
		Results & Success for red line\\ \hline
	\end{tabular}
\end{table}

%UNIT TEST
\begin{table}[H]
	\centering
	\caption{Create Trains' Route}
	\begin{tabular}{|l|l|}
		\hline
		Task & \multicolumn{1}{c|}{Test Design} \\ \hline
		Integrity Level & 3 \\ \hline
		Methods & \parbox[t]{10cm}{Plans a route for a train from one station to another in both MBO \\and FB mode.} \\ \hline
		Inputs &  Number of trains, info from excel file, block occupancy \\ \hline
		Outputs &  Path for a train to take with known authority and speeds. \\ \hline
		Expected Completion & \parbox[t]{10cm}{April 15th}\\ \hline
		Risks and Assumptions & \parbox[t]{10cm}{Able to receive information from excel at load time.} \\ \hline
		Responsibility & MBO\\ \hline
			\\ \hline
		Tested By   &  Zach Scheider\\	\hline
		Date Tested & \parbox[t]{10cm}{April 15th}\\ \hline
		Results & Mostly a success, some information had to be hard coded in.\\ \hline
	\end{tabular}
\end{table}

%UNIT TEST
\begin{table}[H]
	\centering
	\caption{Next occupied}
	\begin{tabular}{|l|l|}
		\hline
		Task & \multicolumn{1}{c|}{Test Design} \\ \hline
		Integrity Level & 3 \\ \hline
		Methods & Finds the next occupied block in a path. \\ \hline
		Inputs &  Block occupancy list, starting position\\ \hline
		Outputs &  First occupied block in the path or null if none are \\ \hline
		Expected Completion & \parbox[t]{10cm}{April 15th}\\ \hline
		Risks and Assumptions & \parbox[t]{10cm}{Receiving correct occupancy information from CTC.} \\ \hline
		Responsibility & MBO\\ \hline
			\\ \hline
		Tested By   &  Zach Scheider\\	\hline
		Date Tested & \parbox[t]{10cm}{April 12th}\\ \hline
		Results & Success\\ \hline
	\end{tabular}
\end{table}

%UNIT TEST
\begin{table}[H]
	\centering
	\caption{Next station}
	\begin{tabular}{|l|l|}
		\hline
		Task & \multicolumn{1}{c|}{Test Design} \\ \hline
		Integrity Level & 3 \\ \hline
		Methods & Finds the next station block in a path. \\ \hline
		Inputs &  Block list, starting position\\ \hline
		Outputs &  First station block in the path or null if none are \\ \hline
		Expected Completion & \parbox[t]{10cm}{April 15th}\\ \hline
		Risks and Assumptions & \parbox[t]{10cm}{Receiving correct station information from dummyTrack.} \\ \hline
		Responsibility & MBO\\ \hline
			\\ \hline
		Tested By   &  Zach Scheider\\	\hline
		Date Tested & \parbox[t]{10cm}{April 15th}\\ \hline
		Results & Success\\ \hline
	\end{tabular}
\end{table}

%UNIT TEST
\begin{table}[H]
	\centering
	\caption{Find authority}
	\begin{tabular}{|l|l|}
		\hline
		Task & \multicolumn{1}{c|}{Test Design} \\ \hline
		Integrity Level & 3 \\ \hline
		Methods & Finds the authority of a train. \\ \hline
		Inputs &  Block list, train to find authority for\\ \hline
		Outputs &  Authority of the given train \\ \hline
		Expected Completion & \parbox[t]{10cm}{April 15th}\\ \hline
		Risks and Assumptions & \parbox[t]{10cm}{Receiving correct station information from dummyTrack, and correct block occupancies from the CTC.} \\ \hline
		Responsibility & MBO\\ \hline
			\\ \hline
		Tested By   &  Zach Scheider\\	\hline
		Date Tested & \parbox[t]{10cm}{April 15th}\\ \hline
		Results & Success\\ \hline
	\end{tabular}
\end{table}

%UNIT TEST
\begin{table}[H]
	\centering
	\caption{Arrive on time}
	\begin{tabular}{|l|l|}
		\hline
		Task & \multicolumn{1}{c|}{Test Design} \\ \hline
		Integrity Level & 2 \\ \hline
		Methods & \parbox[t]{10cm}{Checks to see if a train will arrive at it's destination in time at \\the current speeds.} \\ \hline
		Inputs &  \parbox[t]{10cm}{List of blocks, list of given speeds, current position of the train, \\scheduled time of arrival}\\ \hline
		Outputs &  \parbox[t]{10cm}{Boolean value of whether the train will arrive at it's destination \\on time or not} \\ \hline
		Expected Completion & \parbox[t]{10cm}{April 15th}\\ \hline
		Risks and Assumptions & \parbox[t]{10cm}{Times to the station given in the word document are correct.} \\ \hline
		Responsibility & MBO\\ \hline
			\\ \hline
		Tested By   &  Zach Scheider\\	\hline
		Date Tested & \parbox[t]{10cm}{April 12th}\\ \hline
		Results & Success\\ \hline
	\end{tabular}
\end{table}

%UNIT TEST
\begin{table}[H]
	\centering
	\caption{Calculate Block Speeds}
	\begin{tabular}{|l|l|}
		\hline
		Task & \multicolumn{1}{c|}{Test Design} \\ \hline
		Integrity Level & 2 \\ \hline
		Methods & Calculates the necessary block speeds for a train to arrive on time.\\ \hline
		Inputs &  List of blocks, current position of the train, scheduled time of arrival\\ \hline
		Outputs &  Array of speeds so that the train will arrive in time. \\ \hline
		Expected Completion & \parbox[t]{10cm}{April 15th}\\ \hline
		Risks and Assumptions & \parbox[t]{10cm}{Times to the station given in the word document are correct.} \\ \hline
		Responsibility & MBO\\ \hline
			\\ \hline
		Tested By   &  Zach Scheider\\	\hline
		Date Tested & \parbox[t]{10cm}{April 13th}\\ \hline
		Results & Success\\ \hline
	\end{tabular}
\end{table}

%UNIT TEST
\begin{table}[H]
	\centering
	\caption{Minimum Block Speed}
	\begin{tabular}{|l|l|}
		\hline
		Task & \multicolumn{1}{c|}{Test Design} \\ \hline
		Integrity Level & 2 \\ \hline
		Methods & Calculates the minimum block speed limit in a list of blocks\\ \hline
		Inputs &  List of blocks, starting index\\ \hline
		Outputs &  Lowest block speed limit \\ \hline
		Expected Completion & \parbox[t]{10cm}{April 15th}\\ \hline
		Risks and Assumptions & \parbox[t]{10cm}{Track has the correct block speed limits on them.} \\ \hline
		Responsibility & MBO\\ \hline
			\\ \hline
		Tested By   &  Zach Scheider\\	\hline
		Date Tested & \parbox[t]{10cm}{April 12th}\\ \hline
		Results & Success\\ \hline
	\end{tabular}
\end{table}

%UNIT TEST
\begin{table}[H]
	\centering
	\caption{Block Furthest Below Speed Limit}
	\begin{tabular}{|l|l|}
		\hline
		Task & \multicolumn{1}{c|}{Test Design} \\ \hline
		Integrity Level & 2 \\ \hline
		Methods & Returns the position of the block furthest below it's speed limit\\ \hline
		Inputs &  List of blocks, list of speeds, previous index\\ \hline
		Outputs &  Index the block furthest below it's speed limit \\ \hline
		Expected Completion & \parbox[t]{10cm}{April 15th}\\ \hline
		Risks and Assumptions & \parbox[t]{10cm}{Track has the correct block speed limits on them.} \\ \hline
		Responsibility & MBO\\ \hline
		\\ \hline
		Tested By   &  Zach Scheider\\	\hline
		Date Tested & \parbox[t]{10cm}{April 12th}\\ \hline
		Results & Success\\ \hline
	\end{tabular}
\end{table}

%UNIT TEST
\begin{table}[H]
	\centering
	\caption{Generate Driver Schedule}
	\begin{tabular}{|l|l|}
		\hline
		Task & \multicolumn{1}{c|}{Test Design} \\ \hline
		Integrity Level & 2 \\ \hline
		Methods & Generate a schedule so that a driver occupies each train. \\ \hline
		Inputs &  CSV path \\ \hline
		Outputs &  Driver Schedule\\ \hline
		Expected Completion & \parbox[t]{10cm}{April 15th}\\ \hline
		Risks and Assumptions & \parbox[t]{10cm}{Pulling info from valid CSV file.} \\ \hline
		Responsibility & MBO\\ \hline
		\\ \hline
		Tested By   &  Zach Scheider\\	\hline
		Date Tested & \parbox[t]{10cm}{April 16th}\\ \hline
		Results & Schedule generated but drivers don't get breaks.\\ \hline
	\end{tabular}
\end{table}

\subsection{Integration Tests}

%INTEGRATION TEST
\begin{table}[H]
	\centering
	\caption{Get Mode}
	\begin{tabular}{|l|l|}
		\hline
		Task & \multicolumn{1}{c|}{Test Design} \\ \hline
		Integrity Level & 3 \\ \hline
		Methods & \parbox[t]{10cm}{Receive the correct mode from the CTC, checking for incorrect input. Perform "shutdown operation" if necessary.}\\ \hline
		Inputs &  String of mode \\ \hline
		Outputs &  No outputs \\ \hline
		Expected Completion & \parbox[t]{10cm}{April 15th}\\ \hline
		Risks and Assumptions & \parbox[t]{10cm}{A risk is that communication is interrupted when switching modes.} \\ \hline
		Responsibility & MBO\\ \hline
			\\ \hline
		Tested By   &  Zach Scheider\\	\hline
		Date Tested & \parbox[t]{10cm}{April 19th}\\ \hline
		Results & \parbox[t]{10cm}{Switches to MBO and Manual mode fine. Fixed Block has issues \\some of the time.}\\ \hline
	\end{tabular}
\end{table}

%INTEGRATION TEST
\begin{table}[H]
	\centering
	\caption{Send Information to Train}
	\begin{tabular}{|l|l|}
		\hline
		Task & \multicolumn{1}{c|}{Test Design} \\ \hline
		Integrity Level & 3 \\ \hline
		Methods & \parbox[t]{10cm}{Sends a speed and authority to a train when in MBO mode.}\\ \hline
		Inputs &  Train, suggested speed, authority \\ \hline
		Outputs &  No outputs \\ \hline
		Expected Completion & \parbox[t]{10cm}{April 15th}\\ \hline
		Risks and Assumptions & \parbox[t]{10cm}{A risk is that communication is interrupted.} \\ \hline
		Responsibility & MBO\\ \hline
			\\ \hline
		Tested By   &  Zach Scheider\\	\hline
		Date Tested & \parbox[t]{10cm}{April 18th}\\ \hline
		Results & Success\\ \hline
	\end{tabular}
\end{table}

%INTEGRATION TEST
\begin{table}[H]
	\centering
	\caption{Receive Information from Train}
	\begin{tabular}{|l|l|}
		\hline
		Task & \multicolumn{1}{c|}{Test Design} \\ \hline
		Integrity Level & 2 \\ \hline
		Methods & \parbox[t]{10cm}{Receives actual speed and authority from a train when in MBO mode.}\\ \hline
		Inputs &  Train \\ \hline
		Outputs &  Actual speed and authority \\ \hline
		Expected Completion & \parbox[t]{10cm}{April 15th}\\ \hline
		Risks and Assumptions & \parbox[t]{10cm}{A risk is that communication is interrupted.} \\ \hline
		Responsibility & MBO\\ \hline
		\\ \hline
		Tested By   &  Zach Scheider\\	\hline
		Date Tested & \parbox[t]{10cm}{April 18th}\\ \hline
		Results & Success\\ \hline
	\end{tabular}
\end{table}

%INTEGRATION TEST
\begin{table}[H]
	\centering
	\caption{Send Variance to CTC}
	\begin{tabular}{|l|l|}
		\hline
		Task & \multicolumn{1}{c|}{Test Design} \\ \hline
		Integrity Level & 1 \\ \hline
		Methods & \parbox[t]{10cm}{Sends actual speed and authority to a CTC when in MBO mode.}\\ \hline
		Inputs &  Actual speed and authority \\ \hline
		Outputs &  Actual speed and authority \\ \hline
		Expected Completion & \parbox[t]{10cm}{April 15th}\\ \hline
		Risks and Assumptions & \parbox[t]{10cm}{A risk is that communication is interrupted.} \\ \hline
		Responsibility & MBO\\ \hline
			\\ \hline
		Tested By   &  Zach Scheider\\	\hline
		Date Tested & \parbox[t]{10cm}{April 19th}\\ \hline
		Results & Success\\ \hline
	\end{tabular}
\end{table}

%INTEGRATION TEST
\begin{table}[H]
	\centering
	\caption{Update TrainManager}
	\begin{tabular}{|l|l|}
		\hline
		Task & \multicolumn{1}{c|}{Test Design} \\ \hline
		Integrity Level & 3 \\ \hline
		Methods & \parbox[t]{10cm}{Updates information in TrainManager in MBO and FB mode.}\\ \hline
		Inputs &  TrainManager \\ \hline
		Outputs &  No outputs \\ \hline
		Expected Completion & \parbox[t]{10cm}{April 15th}\\ \hline
		Risks and Assumptions & \parbox[t]{10cm}{Receiving correct information.} \\ \hline
		Responsibility & MBO\\ \hline
			\\ \hline
		Tested By   &  Zach Scheider\\	\hline
		Date Tested & \parbox[t]{10cm}{April 19th}\\ \hline
		Results & Success\\ \hline
	\end{tabular}
\end{table}

%INTEGRATION TEST
\begin{table}[H]
	\centering
	\caption{CTC comm failure}
	\begin{tabular}{|l|l|}
		\hline
		Task & \multicolumn{1}{c|}{Test Design} \\ \hline
		Integrity Level & 1 \\ \hline
		Methods & \parbox[t]{10cm}{Successfully receive failure. Handle failure appropriately. }\\ \hline
		Inputs &  \parbox[t]{10cm}{Failure signal from Murphy. } \\ \hline
		Outputs &  \parbox[t]{10cm}{No outputs. } \\ \hline
		Expected Completion & April 15th \\ \hline
		Risks and Assumptions & \parbox[t]{10cm}{Modules integrated together.}  \\ \hline		
		Responsibility & MBO\\ \hline
			\\ \hline
		Tested By   &  Zach Scheider\\	\hline
		Date Tested & \parbox[t]{10cm}{April 16th}\\ \hline
		Results & FILL IN YOUR RESULTS HERE\\ \hline
	\end{tabular}
\end{table}

%INTEGRATION TEST
\begin{table}[H]
	\centering
	\caption{Train Antenna comm failure}
	\begin{tabular}{|l|l|}
		\hline
		Task & \multicolumn{1}{c|}{Test Design} \\ \hline
		Integrity Level & 1 \\ \hline
		Methods & \parbox[t]{10cm}{Successfully receive failure. Handle failure appropriately. }\\ \hline
		Inputs &  \parbox[t]{10cm}{Failure signal from Murphy. } \\ \hline
		Outputs &  \parbox[t]{10cm}{No outputs. } \\ \hline
		Expected Completion & April 15th \\ \hline
		Risks and Assumptions & \parbox[t]{10cm}{Modules integrated together.}  \\ \hline
		Responsibility & MBO\\ \hline
			\\ \hline
		Tested By   &  Zach Scheider\\	\hline
		Date Tested & \parbox[t]{10cm}{April 16th}\\ \hline
		Results & FILL IN YOUR RESULTS HERE\\ \hline
	\end{tabular}
\end{table}


\section{Changelog}

\begin{table}[H]
	\centering
	\caption{Change}
	\label{changelogl}
	\begin{tabular}{|l|l|}
		\hline
		Date & \multicolumn{1}{c|}{Change} \\ \hline
		March 14, 2017 & General Test Plan \\ \hline
		March 15, 2017 & Add more tests \\ \hline
		April 20, 2017 & Updated test results \\ \hline
	\end{tabular}
\end{table}

\section{Signatures}

\begin{table}[H]
	\centering
	\caption{Signatures}
	\label{signatures}
	\begin{tabular}{|l|l|}
		\hline
		Name, Module & \multicolumn{1}{c|}{Change} \\ \hline
		Zach Scheider, MBO & \\ \hline
	\end{tabular}
\end{table}

\end{document}          
